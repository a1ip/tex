\documentclass{article}
\usepackage{fullpage}
%%% Работа с русским языком
\usepackage[T2A]{fontenc}
\usepackage[utf8]{inputenc}
\usepackage[english,russian]{babel}   %% загружает пакет многоязыковой вёрстки
\usepackage{indentfirst}
\frenchspacing
\usepackage{titlesec} % package to customize chapters, sections and subsections style
%--------------------------------------
\titleformat{\section}{\large\bfseries\centering}{\thesection}{1em}{}
%Hyphenation rules
%--------------------------------------
\usepackage{hyphenat}
\hyphenation{мате-мати-ка восста-навливать}
%--------------------------------------
\usepackage[intlimits,sumlimits]{amsmath}
\usepackage{amsfonts,amssymb,amsthm,mathtools}
\usepackage[bb=pazo,frak=pxtx,frakscaled=1.3]{mathalfa}
\usepackage{upgreek}
\usepackage{graphicx}
\usepackage{verbatim} %% для многострочных комментариев

%% Перенос знаков в формулах (по Львовскому)
\newcommand*{\hm}[1]{#1\nobreak\discretionary{}
{\hbox{$\mathsurround=0pt #1$}}{}}

\renewcommand{\le}{\ensuremath{\leqslant}}
\renewcommand{\leq}{\ensuremath{\leqslant}}
\renewcommand{\ge}{\ensuremath{\geqslant}}
\renewcommand{\geq}{\ensuremath{\geqslant}}

\newcommand{\tg}{\mathop{\rm tg}\nolimits}

\title{Контрольная работа №6 по курсу «Математика»\\
Тема: «Интегральное исчисление функций одной переменной. Ряды.»\\
Вариант №1}
\author{Ригованов Филипп Юрьевич, студент группы КТбз1-1}
\begin{document}
\date{}
\maketitle
\section*{Задание № 1.}

Найти неопределённые интегралы.

$$1)\;\int\left(10+\frac{4}{\sqrt{16-x^2}}-\frac{7}{x^4}-3\sqrt[3]{x^2}+\cos{x}\right)\,dx;\qquad2)\;\int\sqrt[3]{5x+4}\,dx;$$

$$3)\;\int\frac{dx}{x+2x^{5/7}};\qquad4)\;\int x\cdot\arctg{x}\,dx;\qquad5)\;\int\frac{(x+3)dx}{x^2+3x+2}.$$

\begin{center}Решение:\end{center}

\section*{Задание № 2.}

Найти неопределённый интеграл.

$$\int\limits_{0}^{\pi/2}\frac{dx}{4-\sin{x}+\cos{x}}.$$

\begin{center}Решение:\end{center}

\section*{Задание № 3.}

Исследовать на сходимость несобственный интеграл.

$$\int\limits_{0}^{+\infty}\frac{x+\sin^2{x}}{x^2+3}\,dx.$$

\begin{center}Решение:\end{center}

\section*{Задание № 4.}

Исследовать на сходимость числовые ряды $\sum\limits_{n=1}^{\infty}a_n$.

$$1)\;a_n=\frac{n+10}{n^2+2};\qquad2)\;a_n=\frac{n^2+2}{(n+1)!};\qquad3)\;a_n=\frac{1}{(\ln{(n+1)})^n}\;.$$

\begin{center}Решение:\end{center}

\section*{Задание № 5.}

Исследовать ряд на абсолютную и условную сходимость.

$$\sum\limits_{n=1}^{\infty}(-1)^{n-1}\cdot\frac{3^n}{(5n+2)^n}.$$

\begin{center}Решение:\end{center}

\section*{Задание № 6.}

Найти область сходимости степенного ряда.

$$\sum\limits_{n=1}^{\infty}\frac{(x-1)^n}{(3n-1)^2\cdot3^n}.$$

\begin{center}Решение:\end{center}

\section*{Задание № 7.}

Разложить в тригонометрический ряд Фурье функцию $f(x)$.

\begin{equation*}
f(x)=
 \begin{cases}
  1, & \textit{если}\quad -1<x\leq0,\\
  2-x, & \textit{если}\quad 0<x<1.
 \end{cases}
\end{equation*}

\begin{center}Решение:\end{center}

\end{document}
