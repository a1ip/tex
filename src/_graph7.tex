\documentclass[fleqn]{article}
\usepackage{fullpage}
%%% Работа с русским языком
\usepackage[T2A]{fontenc}
\usepackage[utf8]{inputenc}
\usepackage[english,russian]{babel}   %% загружает пакет многоязыковой вёрстки
\usepackage{indentfirst}
\frenchspacing
\usepackage{titlesec} % package to customize chapters, sections and subsections style
%--------------------------------------
\titleformat{\section}{\large\bfseries\centering}{\thesection}{1em}{}
%Hyphenation rules
%--------------------------------------
\usepackage{hyphenat}
\hyphenation{мате-мати-ка восста-навливать}
%--------------------------------------
\usepackage[intlimits,sumlimits,fleqn]{amsmath}
\usepackage{amsfonts,amssymb,amsthm,mathtools}
\usepackage[bb=pazo,frak=pxtx,frakscaled=1.3]{mathalfa}
\usepackage{upgreek}
\usepackage{graphicx}
\usepackage{verbatim} %% для многострочных комментариев

%% Перенос знаков в формулах (по Львовскому)
\newcommand*{\hm}[1]{#1\nobreak\discretionary{}
{\hbox{$\mathsurround=0pt #1$}}{}}

\renewcommand{\le}{\ensuremath{\leqslant}}
\renewcommand{\leq}{\ensuremath{\leqslant}}
\renewcommand{\ge}{\ensuremath{\geqslant}}
\renewcommand{\geq}{\ensuremath{\geqslant}}

\newcommand{\tg}{\mathop{\rm tg}\nolimits}

\usepackage{tikz}
\usepackage{pgfplots}
\pgfplotsset{width=10cm,compat=1.9}
\usepgfplotslibrary{external}
\tikzexternalize
\tikzset{>=latex}

\usetikzlibrary{arrows.meta}

\usepackage{xcolor,colortbl}

\newcommand{\mc}[2]{\multicolumn{#1}{c}{#2}}
\definecolor{Gray}{gray}{0.85}

\newcolumntype{a}{>{\columncolor{Gray}}c}
\newcolumntype{b}{>{\columncolor{white}}c}

\DeclareMathOperator{\sign}{sign}

\title{Контрольная работа по курсу «Теория графов»\\
Вариант №7}

\begin{document}
\date{}
\maketitle
\section*{Задание № 1.}

По матрице смежности восстановите ориентированный граф $D$,
взяв в качестве вершин $v_1$, $v_2$, $v_3$, $v_4$, $v_5$ пять произвольных точек плоскости. Найдите:

1) матрицу инцидентности $B$, предварительно перенумеровав ребра;

2) матрицу достижимости $T$;

3) матрицу сильной связности;

4) компоненты сильной связности.

$$A = \begin{pmatrix}
0 & 1 & 0 & 0 & 1 \\
0 & 0 & 0 & 1 & 0 \\
0 & 1 & 0 & 0 & 0 \\
1 & 0 & 1 & 0 & 1 \\
0 & 0 & 0 & 0 & 0 \\
\end{pmatrix}
$$

\begin{center}Решение:\end{center}

Сначала восстановим ориентированный граф и пронумеруем его ребра.

\medskip

\begin{tikzpicture}
\begin{scope}[every node/.style={circle,thick,draw}]
    \node (A) at (-2,2) {$v_1$};
    \node (B) at (2,2) {$v_2$};
    \node (C) at (2,-2) {$v_3$};
    \node (D) at (0,0) {$v_4$};
    \node (E) at (-2,-2) {$v_5$};
\end{scope}

\begin{scope}[>={Stealth[black]},
              every node/.style={fill=white,circle},
              every edge/.style={draw=red,very thick}]
    \path [->] (A) edge node {$x_1$} (B);
    \path [->] (A) edge node {$x_7$} (E);
    \path [->] (B) edge node {$x_2$} (D);
    \path [->] (C) edge node {$x_3$} (B);
    \path [->] (D) edge node {$x_4$} (A);
    \path [->] (D) edge node {$x_5$} (C);
    \path [->] (D) edge node {$x_6$} (E);
\end{scope}
\end{tikzpicture}

1) Тогда матрицей инцидентности ориентированного графа $D$ будет

$$B(D)=
\begin{pmatrix}
-1 & 0 & 0 & 1 & 0 & 0 & -1 \\
1 & -1 & 1 & 0 & 0 & 0 & 0 \\
0 & 0 & -1 & 0 & 1 & 0 & 0 \\
0 & 1 & 0 & -1 & -1 & -1 & 0 \\
0 & 0 & 0 & 0 & 0 & 1 & 1 \\
\end{pmatrix}
$$
\newpage
2) Найдем матрицу достижимости ориентированного графа $D$:

$$
\sign{A^2} = \begin{pmatrix}
0 & 0 & 0 & 1 & 0 \\
1 & 0 & 1 & 0 & 1 \\
0 & 0 & 0 & 1 & 0 \\
0 & 2 & 0 & 0 & 1 \\
0 & 0 & 0 & 0 & 0 \\
\end{pmatrix}
;\quad
\sign{A^3} = \begin{pmatrix}
1 & 0 & 1 & 0 & 1 \\
0 & 2 & 0 & 0 & 1 \\
1 & 0 & 1 & 0 & 1 \\
0 & 0 & 0 & 2 & 0 \\
0 & 0 & 0 & 0 & 0 \\
\end{pmatrix}
;\quad
\sign{A^4} = \begin{pmatrix}
0 & 2 & 0 & 0 & 1 \\
0 & 0 & 0 & 2 & 0 \\
0 & 2 & 0 & 0 & 1 \\
2 & 0 & 2 & 0 & 2 \\
0 & 0 & 0 & 0 & 0 \\
\end{pmatrix}.
$$

$T(D)= \sign{[E + A + A^2 + A^3 + A^4]}=$
$$=\sign\left[\begin{pmatrix}
1 & 0 & 0 & 0 & 0\\
0 & 1 & 0 & 0 & 0\\
0 & 0 & 1 & 0 & 0\\
0 & 0 & 0 & 1 & 0\\
0 & 0 & 0 & 0 & 1\\
\end{pmatrix}
+
\begin{pmatrix}
0 & 1 & 0 & 0 & 1 \\
0 & 0 & 0 & 1 & 0 \\
0 & 1 & 0 & 0 & 0 \\
1 & 0 & 1 & 0 & 1 \\
0 & 0 & 0 & 0 & 0 \\
\end{pmatrix}
+
\begin{pmatrix}
0 & 0 & 0 & 1 & 0 \\
1 & 0 & 1 & 0 & 1 \\
0 & 0 & 0 & 1 & 0 \\
0 & 2 & 0 & 0 & 1 \\
0 & 0 & 0 & 0 & 0 \\
\end{pmatrix}
+
\begin{pmatrix}
1 & 0 & 1 & 0 & 1 \\
0 & 2 & 0 & 0 & 1 \\
1 & 0 & 1 & 0 & 1 \\
0 & 0 & 0 & 2 & 0 \\
0 & 0 & 0 & 0 & 0 \\
\end{pmatrix}
+
\begin{pmatrix}
0 & 2 & 0 & 0 & 1 \\
0 & 0 & 0 & 2 & 0 \\
0 & 2 & 0 & 0 & 1 \\
2 & 0 & 2 & 0 & 2 \\
0 & 0 & 0 & 0 & 0 \\
\end{pmatrix}\right]
=$$
$$=\begin{pmatrix}
1 & 1 & 1 & 1 & 1\\
1 & 1 & 1 & 1 & 1\\
1 & 1 & 1 & 1 & 1\\
1 & 1 & 1 & 1 & 1\\
0 & 0 & 0 & 0 & 1\\
\end{pmatrix}
$$

3) Найдем матрицу сильной связности ориентированного графа $D$:

$$S(D)= T(D) \:\&\: T^T(D)=
\begin{pmatrix}
1 & 1 & 1 & 1 & 1\\
1 & 1 & 1 & 1 & 1\\
1 & 1 & 1 & 1 & 1\\
1 & 1 & 1 & 1 & 1\\
0 & 0 & 0 & 0 & 1\\
\end{pmatrix}
\:\&\:
\begin{pmatrix}
1 & 1 & 1 & 1 & 0\\
1 & 1 & 1 & 1 & 0\\
1 & 1 & 1 & 1 & 0\\
1 & 1 & 1 & 1 & 0\\
1 & 1 & 1 & 1 & 1\\
\end{pmatrix}
=\begin{pmatrix}
1 & 1 & 1 & 1 & 0\\
1 & 1 & 1 & 1 & 0\\
1 & 1 & 1 & 1 & 0\\
1 & 1 & 1 & 1 & 0\\
0 & 0 & 0 & 0 & 1\\
\end{pmatrix}.
$$

4) Выделим компоненты связности ориентированного графа $D$.

Присваиваем $p=1$, $S_1 = S(D)$ и составляем множество вершин
первой компоненты сильной связности $D_1$: это те вершины, которым соответствуют единицы в первой строке матрицы $S(D)$. Таким образом, первая компонента сильной связности состоит из четырёх вершин $V_1=\{v_1, v_2, v_3, v_4\}$. Составляем матрицу смежности для компоненты сильной связности $D_1$ исходного графа $D$. Для этого возьмем подматрицу матрицы $A(D)$, состоящую из элементов матрицы $A$, находящихся на пересечении строк и столбцов, соответствующих вершинам из $V_1$:

$$A(D_1)=
\begin{pmatrix}
0 & 1 & 0 & 0\\
0 & 0 & 0 & 1\\
0 & 1 & 0 & 0\\
1 & 0 & 1 & 0\\
\end{pmatrix}
$$

Вычеркиваем из матрицы $S_1(D)$ строки и столбцы,
соответствующие вершинам $v_1$, $v_2$, $v_3$ и $v_4$, чтобы получить матрицу $S_2(D)$, которая состоит из одного элемента: $S_2(D)=(1)$
и присваиваем $p=2$. Значит вторая компонента сильной связности исходного графа, состоит из одной вершины $V_2 = \{v_5\}$.

Таким образом, выделены $2$ компоненты сильной связности ориентированного графа $D$:

\medskip

\begin{tikzpicture}
\begin{scope}
    \node (D1) at (-2,3) {$D_1:$};
    \node (D2) at (5,3) {$D_2:$};
\end{scope}
\begin{scope}[every node/.style={circle,thick,draw}]
    \node (A) at (-2,2) {$v_1$};
    \node (B) at (2,2) {$v_2$};
    \node (C) at (2,-2) {$v_3$};
    \node (D) at (0,0) {$v_4$};
    \node (E) at (5,2) {$v_5$};
\end{scope}

\begin{scope}[>={Stealth[black]},
              every node/.style={fill=white,circle},
              every edge/.style={draw=red,very thick}]
    \path [->] (A) edge node {$x_1$} (B);
    \path [->] (B) edge node {$x_2$} (D);
    \path [->] (C) edge node {$x_3$} (B);
    \path [->] (D) edge node {$x_4$} (A);
    \path [->] (D) edge node {$x_5$} (C);
\end{scope}
\end{tikzpicture}

\section*{Задание № 2.}

Дан ориентированный граф $D$.

1. Найдите матрицу смежности $A$.

2. С помощью алгоритма фронта волны найдите расстояния в
ориентированном графе $D$, диаметр, радиус и центры.

\medskip

\begin{tikzpicture}
\begin{scope}[every node/.style={circle,thick,draw}]
    \node (A) at (-2.5,1) {$v_1$};
    \node (B) at (0,3) {$v_2$};
    \node (C) at (2.5,1) {$v_3$};
    \node (D) at (1.8,-2.2) {$v_4$};
    \node (E) at (-1.8,-2.2) {$v_5$};
    \node (F) at (0,0) {$v_6$};
\end{scope}

\begin{scope}[>={Stealth[black]},
              every node/.style={fill=white,circle},
              every edge/.style={draw=red,very thick}]
    \path [->] (A) edge (B);
    \path [->] (A) edge (D);
    \path [->] (B) edge (C);
    \path [->] (B) edge (E);
    \path [->] (C) edge (A);
    \path [->] (C) edge (F);
    \path [->] (D) edge (C);
    \path [->] (D) edge (F);
    \path [->] (E) edge (A);
    \path [->] (E) edge (D);
    \path [->] (F) edge (B);
    \path [->] (F) edge (E);
\end{scope}
\end{tikzpicture}

\begin{center}Решение:\end{center}

1) Матрица смежности ориентированного графа $D$ имеет вид:

$$A(D)=
\begin{pmatrix}
0 & 1 & 0 & 1 & 0 & 0 \\
0 & 0 & 1 & 0 & 1 & 0 \\
1 & 0 & 0 & 0 & 0 & 1 \\
0 & 0 & 1 & 0 & 0 & 1 \\
1 & 0 & 0 & 1 & 0 & 0 \\
0 & 1 & 0 & 0 & 1 & 0 \\
\end{pmatrix}
$$

2) Заполним матрицу минимальных расстояний, сперва поставив нули по главной диагонали, затем перенеся единицы из матрицы смежности, и после этого используя алгоритм фронта волны для каждой из оставшихся пар вершин графа $D$. А именно: фиксикуем строку, смотрим в какие неотмеченные вершины мы можем попасть из единичек в данной строке за один шаг и заполняем их двойками. Так заполняем двойками все строки. И, нам очень повезло, на этом шаге матрица минимальных расстояний у нас заполнится.

$$R(D)=
\begin{pmatrix}
0 & 1 & 2 & 1 & 2 & 2 \\
2 & 0 & 1 & 2 & 1 & 2 \\
1 & 2 & 0 & 2 & 2 & 1 \\
2 & 2 & 1 & 0 & 2 & 1 \\
1 & 2 & 2 & 1 & 0 & 2 \\
2 & 1 & 2 & 2 & 1 & 0 \\
\end{pmatrix}
$$

Из матрицы $R(D)$ минимальных расстояний находим диаметр:

$$d(D)=\max\limits_{v,w\in V} d(v,w)=r_{13}=2.$$

Для каждой вершины ориентированного графа $D$ найдем эксцентриситет (максимальное удаление от вершины) по формуле $r(v)=\max\limits_{w\in V} d(v,w)$:

$$r(v_1)=2,\quad r(v_2)=2,\quad r(v_3)=2,\quad r(v_4)=2,\quad r(v_5)=2,\quad r(v_6)=2.$$

Значит радиусом графа $D$ будет:

$$r(D)=\min\limits_{v\in V}r(v)=2.$$

Центрами графа будут являться все вершины графа, так как все они они имеют одинаковый (он же минимальный) эксцентриситет.

\section*{Задание № 3.}

Найдите минимальный путь в нагруженном ориентированном графе $D$ по методу Форда-Беллмана из указанной вершины в указанную.

Из $v_2$ в $v_4$.

\medskip

\begin{tikzpicture}
\begin{scope}[every node/.style={circle,thick,draw}]
    \node (A) at (-4,0) {$v_1$};
    \node (B) at (-2,3) {$v_2$};
    \node (C) at (2,3) {$v_3$};
    \node (D) at (4,0) {$v_4$};
    \node (E) at (2,-3) {$v_5$};
    \node (F) at (-2,-3) {$v_6$};
    \node (G) at (0,0) {$v_7$};
\end{scope}

\begin{scope}[>={Stealth[black]},
              every node/.style={fill=white,circle},
              every edge/.style={draw=red,very thick}]
    \path [->] (A) edge node[pos=0.25] {$10$} (C);
    \path [->] (A) edge node {$4$} (G);
    \path [->] (B) edge node {$1$} (A);
    \path [->] (B) edge node {$12$} (C);
    \path [->] (C) edge node {$3$} (D);
    \path [->] (C) edge node {$9$} (G);
    \path [->] (D) edge node {$8$} (G);
    \path [->] (D) edge[bend left=15] node {$3$} (E);
    \path [->] (E) edge[bend left=15] node {$6$} (D);
    \path [->] (E) edge[bend left=15] node {$9$} (F);
    \path [->] (F) edge[bend left=15] node {$7$} (E);
    \path [->] (G) edge node[pos=0.25] {$8$} (B);
    \path [->] (G) edge node {$8$} (E);
    \path [->] (F) edge[bend left=15] node {$6$} (G);
    \path [->] (G) edge[bend left=15] node {$5$} (F);
    \path [->] (F) edge node {$5$} (A);
\end{scope}
\end{tikzpicture}

\begin{center}Решение:\end{center}

Матрица длин дуг для нагруженного ориентированного графа $D$ будет следующей:
$$C(D)=
\begin{pmatrix}
\infty & \infty & 10 & \infty & \infty & \infty & 4 \\
1 & \infty & 12 & \infty & \infty & \infty & \infty \\
\infty & \infty & \infty & 3 & \infty & \infty & 9 \\
\infty & \infty & \infty & \infty & 3 & \infty & 8 \\
\infty & \infty & \infty & 6 & \infty & 9 & \infty \\
5 & \infty & \infty & \infty & 7 & \infty & 6 \\
\infty & 8 & \infty & \infty & 8 & 5 & \infty \\
\end{pmatrix}
$$

Согласно алгоритму метода Форда-Беллмана, составляем таблицу стоимости минимальных путей из вершины $v_2$ в вершину $v_i$ не более, чем за $k$ шагов. Заполняем вторую строку нулями, а элементы первого столбца, кроме второго, заполняем $\infty$. Второй столбец, кроме второго элемента, заполняем второй строкой матрицы длин дуг $C(D)$. Далее каждый следующий столбец получаем находя минимум чисел полученых сложением соответствующих элементов предыдущего столбца таблицы с элементами соответствующего столбца матрицы длин дуг $C(D)$.

\medskip
\bgroup
\def\arraystretch{1.5}
\setlength\tabcolsep{6}
\begin{tabular}{|>{\columncolor{Gray}}c|c|c|c|c|c|c|c|}
\hline
\rowcolor{Gray}
\cellcolor{white} & $\lambda_i^{(0)}$ & $\lambda_i^{(1)}$ & $\lambda_i^{(2)}$ & $\lambda_i^{(3)}$ & $\lambda_i^{(4)}$ & $\lambda_i^{(5)}$ & $\lambda_i^{(6)}$ \\
\hline
$v_1$ & $\infty$ & 1 & 1 & 1 & 1 & 1 & 1 \\
\hline
$v_2$ & 0 & 0 & 0 & 0 & 0 & 0 & 0 \\
\hline
$v_3$ & $\infty$ & 12 & 11 & 11 & 11 & 11 & 11 \\
\hline
$v_4$ & $\infty$ & $\infty$ & 15 & 14 & 14 & 14 & 14 \\
\hline
$v_5$ & $\infty$ & $\infty$ & $\infty$ & 13 & 13 & 13 & 13 \\
\hline
$v_6$ & $\infty$ & $\infty$ & $\infty$ & 10 & 10 & 10 & 10 \\
\hline
$v_7$ & $\infty$ & $\infty$ & 5 & 5 & 5 & 5 & 5 \\
\hline
\end{tabular}
\egroup
\medskip

$\lambda_1^{(2)} =\min\{1+\infty,\; 0+1,\; 12+\infty,\; \infty+\infty,\; \infty+\infty,\; \infty+5,\; \infty+\infty\}=1.$

$\lambda_3^{(2)} =\min\{1+10,\; 0+12,\; 12+\infty,\; \infty+\infty,\; \infty+\infty,\; \infty+\infty,\; \infty+\infty\}=11.$

$\lambda_4^{(2)} =\min\{1+\infty,\; 0+\infty,\; 12+3,\; \infty+\infty,\; \infty+6,\; \infty+\infty,\; \infty+\infty\}=15.$

$\lambda_5^{(2)} =\min\{1+\infty,\; 0+\infty,\; 12+\infty,\; \infty+3,\; \infty+\infty,\; \infty+7,\; \infty+8\}=\infty.$

$\lambda_6^{(2)} =\min\{1+\infty,\; 0+\infty,\; 12+\infty,\; \infty+\infty,\; \infty+9,\; \infty+\infty,\; \infty+5\}=\infty.$

$\lambda_7^{(2)} =\min\{1+4,\; 0+\infty,\; 12+9,\; \infty+8,\; \infty+\infty,\; \infty+6,\; \infty+\infty\}=5.$

\medskip

$\lambda_1^{(3)} =\min\{1+\infty,\; 0+1,\; 11+\infty,\; 15+\infty,\; \infty+\infty,\; \infty+5,\; 5+\infty\}=1.$

$\lambda_3^{(3)} =\min\{1+10,\; 0+12,\; 11+\infty,\; 15+\infty,\; \infty+\infty,\; \infty+\infty,\; 5+\infty\}=11.$

$\lambda_4^{(3)} =\min\{1+\infty,\; 0+\infty,\; 11+3,\; 15+\infty,\; \infty+6,\; \infty+\infty,\; 5+\infty\}=14.$

$\lambda_5^{(3)} =\min\{1+\infty,\; 0+\infty,\; 11+\infty,\; 15+3,\; \infty+\infty,\; \infty+7,\; 5+8\}=13.$

$\lambda_6^{(3)} =\min\{1+\infty,\; 0+\infty,\; 11+\infty,\; 15+\infty,\; \infty+9,\; \infty+\infty,\; 5+5\}=10.$

$\lambda_7^{(3)} =\min\{1+4,\; 0+\infty,\; 11+9,\; 15+8,\; \infty+\infty,\; \infty+6,\; 5+\infty\}=5.$

\medskip

$\lambda_1^{(4)} =\min\{1+\infty,\; 0+1,\; 11+\infty,\; 14+\infty,\; 13+\infty,\; 10+5,\; 5+\infty\}=1.$

$\lambda_3^{(4)} =\min\{1+10,\; 0+12,\; 11+\infty,\; 14+\infty,\; 13+\infty,\; 10+\infty,\; 5+\infty\}=11.$

$\lambda_4^{(4)} =\min\{1+\infty,\; 0+\infty,\; 11+3,\; 14+\infty,\; 13+6,\; 10+\infty,\; 5+\infty\}=14.$

$\lambda_5^{(4)} =\min\{1+\infty,\; 0+\infty,\; 11+\infty,\; 14+3,\; 13+\infty,\; 10+7,\; 5+8\}=13.$

$\lambda_6^{(4)} =\min\{1+\infty,\; 0+\infty,\; 11+\infty,\; 14+\infty,\; 13+9,\; 10+\infty,\; 5+5\}=10.$

$\lambda_7^{(4)} =\min\{1+4,\; 0+\infty,\; 11+9,\; 14+8,\; 13+\infty,\; 10+6,\; 5+\infty\}=5.$

\medskip

Следующие два столбца $\lambda_i^{(5)}$ и $\lambda_i^{(6)}$ вычисляются полностью идентично предыдущему.

Теперь необходимо по построенной таблице и по матрице $C(D)$ восстановить минимальный путь из вершины $v_2$ в $v_4$, который будет строиться с конца, то есть начиная с вершины $v_4$.

Для этого выбираем минимальное из чисел, стоящих в строке, соответствующей $v_4$ в таблице. Это число – $14$ является длиной минимального искомого пути. В первый раз такая длина была получена при количестве шагов, равном $3$.

В вершину $v_4$ мы можем попасть за один шаг из вершин $v_3$ и $v_5$ (длина соответствующих дуг $3$ и $6$ соответственно – данные из матрицы $C(D)$). Выбираем минимальную по длине из этих дуг. Далее, в вершину $v_3$  из вершины $v_2$ можно за два хода попасть только через вершину $v_1$.

Таким образом искомый путь за $3$ шага минимальной длины 14 из вершины $v_2$ в вершину $v_4$ найден: $\langle v_2, v_1, v_3, v_4\rangle$.

\section*{Задание № 4.}

Найдите минимальное остовное дерево в неориентированном нагруженном графе.

\medskip

\begin{tikzpicture}
\begin{scope}[every node/.style={circle,thick,draw}]
    \node (A) at (-4,0) {$v_1$};
    \node (B) at (-2,3) {$v_2$};
    \node (C) at (2,3) {$v_3$};
    \node (D) at (4,0) {$v_4$};
    \node (E) at (2,-3) {$v_5$};
    \node (F) at (-2,-3) {$v_6$};
    \node (G) at (0,1) {$v_7$};
\end{scope}

\begin{scope}[-,
              every node/.style={fill=white,circle},
              every edge/.style={draw=red,very thick}]
    \path (A) edge node {$3$} (B);
    \path (A) edge node[pos=0.75] {$12$} (C);
    \path (A) edge node[pos=0.3] {$2$} (D);
    \path (A) edge node {$20$} (F);
    \path (G) edge node[pos=0.25] {$6$} (A);
    \path (G) edge node[pos=0.65] {$3$} (B);
    \path (G) edge node {$5$} (C);
    \path (G) edge node[pos=0.25] {$7$} (D);
    \path (G) edge node[pos=0.45] {$3$} (E);
    \path (G) edge node {$2$} (F);
    \path (B) edge node {$1$} (C);
    \path (C) edge node {$8$} (D);
    \path (D) edge node {$9$} (E);
    \path (E) edge node {$10$} (F);
    \path (D) edge node[pos=0.75] {$7$} (F);
    \path (B) edge node[pos=0.75] {$4$} (F);
    \path (C) edge node[pos=0.25] {$12$} (E);
\end{scope}
\end{tikzpicture}

\begin{center}Решение:\end{center}

$$
C(G)=
\begin{pmatrix}
0 & 3 & 12 & 2 & \infty & 20 & 6 \\
3 & 0 & 1 & \infty & \infty & 4 & 3 \\
12 & 1 & 0 & 8 & 12 & \infty & 5 \\
2 & \infty & 8 & 0 & 9 & 7 & 7 \\
\infty & \infty & 12 & 9 & 0 & 10 & 3 \\
20 & 4 & \infty & 7 & 10 & 0 & 2 \\
6 & 3 & 5 & 7 & 3 & 2 & 0 \\
\end{pmatrix}
$$

$G_2=(\{v_2,v_3\},\{x_{23}\})$,\quad $l(G_2)=c_{23}=1$;

$G_3=(\{v_1,v_2,v_3\},\{x_{12},x_{23}\})$,\quad $l(G_3)=l(G_2)+c_{12}=1+3=4$;

$G_4=(\{v_4,v_1,v_2,v_3\},\{x_{41},x_{12},x_{23}\})$,\quad $l(G_4)=l(G_3)+c_{41}=4+2=6$;

$G_5=(\{v_7,v_4,v_1,v_2,v_3\},\{x_{27},x_{41},x_{12},x_{23}\})$,\quad $l(G_5)=l(G_4)+c_{27}=6+3=9$;

$G_6=(\{v_7,v_4,v_1,v_2,v_3,v_6\},\{x_{27},x_{41},x_{12},x_{23},x_{76}\})$,\quad $l(G_6)=l(G_5)+c_{76}=9+2=11$;

$G_7=(\{v_7,v_4,v_1,v_2,v_3,v_6,v_5\},\{x_{27},x_{41},x_{12},x_{23},x_{76},x_{75}\})$,\quad $l(G_7)=l(G_6)+c_{75}=11+3=14$;

Поскольку $i=7=n(G)$, то работа алгоритма на этом заканчивается.

Таким образом, искомое минимальное остовное дерево графа $G$ --− это граф $G_7$, изображенный ниже, длина которого равна $14$.

\medskip

\begin{tikzpicture}
\begin{scope}[every node/.style={circle,thick,draw}]
    \node (A) at (-4,0) {$v_1$};
    \node (B) at (-2,3) {$v_2$};
    \node (C) at (2,3) {$v_3$};
    \node (D) at (4,0) {$v_4$};
    \node (E) at (2,-3) {$v_5$};
    \node (F) at (-2,-3) {$v_6$};
    \node (G) at (0,1) {$v_7$};
\end{scope}

\begin{scope}[-,
              every node/.style={fill=white,circle},
              every edge/.style={draw=red,very thick}]
    \path (B) edge node {$1$} (C);
    \path (D) edge node[pos=0.75] {$2$} (A);
    \path (B) edge node {$3$} (A);
    \path (B) edge node {$3$} (G);
    \path (G) edge node {$2$} (F);
    \path (G) edge node {$3$} (E);
\end{scope}
\end{tikzpicture}

\section*{Задание № 5.}

Методом ветвей и границ найдите оптимальный путь коммивояжёра при следующей матрице стоимости.

\medskip

\begin{tabular}{|>{\columncolor{Gray}}c|c|c|c|c|c|}
\hline
\rowcolor{Gray}
\cellcolor{white} & 1 & 2 & 3 & 4 & 5 \\
\hline
1 & $\infty$ & 7 & 3 & 3 & 2 \\
\hline
2 & 7 & $\infty$ & 6 & 9 & 3 \\
\hline
3 & 3 & 6 & $\infty$ & 7 & 12 \\
\hline
4 & 3 & 9 & 7 & $\infty$ & 4 \\
\hline
5 & 2 & 3 & 12 & 4 & $\infty$ \\
\hline
\end{tabular}

\begin{center}Решение:\end{center}

Приведём матрицу по строкам:

\medskip
\begin{tabular}{|>{\columncolor{Gray}}c|c|c|c|c|c|}
\hline
\rowcolor{Gray}
\cellcolor{white} & 1 & 2 & 3 & 4 & 5 \\
\hline
1 & $\infty$ & 5 & 1 & 1 & 0 \\
\hline
2 & 4 & $\infty$ & 3 & 6 & 0 \\
\hline
3 & 0 & 3 & $\infty$ & 4 & 9 \\
\hline
4 & 0 & 6 & 4 & $\infty$ & 1 \\
\hline
5 & 0 & 1 & 10 & 2 & $\infty$ \\
\hline
\end{tabular}
\medskip

Затем по столбцам:

\medskip
\begin{tabular}{|>{\columncolor{Gray}}c|c|c|c|c|c|}
\hline
\rowcolor{Gray}
\cellcolor{white} & 1 & 2 & 3 & 4 & 5 \\
\hline
1 & $\infty$ & 4 & 0 & 0 & 0 \\
\hline
2 & 4 & $\infty$ & 2 & 5 & 0 \\
\hline
3 & 0 & 2 & $\infty$ & 3 & 9 \\
\hline
4 & 0 & 5 & 3 & $\infty$ & 1 \\
\hline
5 & 0 & 0 & 9 & 1 & $\infty$ \\
\hline
\end{tabular}
\medskip

Обозначим полученную выше матрицу за $C_1$.

Сумма констант приведения $\varphi\left(\Gamma\right)=2+3+3+3+2+1+1+1=16$.

Найдём веса нулей в матрице $C_1$:

\medskip
\begin{tabular}{|>{\columncolor{Gray}}c|c|c|c|c|c|}
\hline
\rowcolor{Gray}
\cellcolor{white} & 1 & 2 & 3 & 4 & 5 \\
\hline
1 & $\infty$ & 4 & $0_{(2)}$ & $0_{(1)}$ & $0_{(0)}$ \\
\hline
2 & 4 & $\infty$ & 2 & 5 & $0_{(2)}$ \\
\hline
3 & $0_{(2)}$ & 2 & $\infty$ & 3 & 9 \\
\hline
4 & $0_{(1)}$ & 5 & 3 & $\infty$ & 1 \\
\hline
5 & $0_{(0)}$ & $0_{(2)}$ & 9 & 1 & $\infty$ \\
\hline
\end{tabular}
\medskip

Самыми тяжёлыми оказались нули в клетках $(5,2)$, $(2,5)$, $(1,3)$ и $(3,1)$. Мы можем выбрать любую из них, пусть это будет клетка $(5,2)$.

Матрицу $C_{1,1}$, которая соответстует множеству $\Gamma_{\{(5,2)\}}$, получим вычеркиванием второго столбца и пятой строки из матрицы $C_1$, причём число в клетке $(2,5)$ нужно заменить на $\infty$, чтобы не получалось коротких циклов:

\medskip
\begin{tabular}{|>{\columncolor{Gray}}c|c|c|c|c|}
\hline
\rowcolor{Gray}
\cellcolor{white} & 1 & 3 & 4 & 5 \\
\hline
1 & $\infty$ & 0 & 0 & 0 \\
\hline
2 & 4 & 2 & 5 & $\infty$ \\
\hline
3 & 0 & $\infty$ & 3 & 9 \\
\hline
4 & 0 & 3 & $\infty$ & 1 \\
\hline
\end{tabular}
\medskip

Приведем матрицу по строкам:

\medskip
\begin{tabular}{|>{\columncolor{Gray}}c|c|c|c|c|}
\hline
\rowcolor{Gray}
\cellcolor{white} & 1 & 3 & 4 & 5 \\
\hline
1 & $\infty$ & 0 & 0 & 0 \\
\hline
2 & 2 & 0 & 3 & $\infty$ \\
\hline
3 & 0 & $\infty$ & 3 & 9 \\
\hline
4 & 0 & 3 & $\infty$ & 1 \\
\hline
\end{tabular}
\medskip

По столбцам матрица $C_{1,1}$ уже приведённая.

$\varphi\left(\Gamma_{\{(5,2)\}}\right)=\varphi\left(\Gamma\right)+2=16+2=18$.

Множеству же $\Gamma_{\{\overline{(5,2)}\}}$, соответствует матрица $C_{1,2}$, полученная заменой на $\infty$ элемента $(5,2)$ в матрице $C_1$:

\medskip
\begin{tabular}{|>{\columncolor{Gray}}c|c|c|c|c|c|}
\hline
\rowcolor{Gray}
\cellcolor{white} & 1 & 2 & 3 & 4 & 5 \\
\hline
1 & $\infty$ & 4 & 0 & 0 & 0 \\
\hline
2 & 4 & $\infty$ & 2 & 5 & 0 \\
\hline
3 & 0 & 2 & $\infty$ & 3 & 9 \\
\hline
4 & 0 & 5 & 3 & $\infty$ & 1 \\
\hline
5 & 0 & $\infty$ & 9 & 1 & $\infty$ \\
\hline
\end{tabular}
\medskip

По строкам приведение не требуется, а после приведения по столбцам матрица $C_{1,2}$ имеет вид:

\medskip
\begin{tabular}{|>{\columncolor{Gray}}c|c|c|c|c|c|}
\hline
\rowcolor{Gray}
\cellcolor{white} & 1 & 2 & 3 & 4 & 5 \\
\hline
1 & $\infty$ & 2 & 0 & 0 & 0 \\
\hline
2 & 4 & $\infty$ & 2 & 5 & 0 \\
\hline
3 & 0 & 0 & $\infty$ & 3 & 9 \\
\hline
4 & 0 & 3 & 3 & $\infty$ & 1 \\
\hline
5 & 0 & $\infty$ & 9 & 1 & $\infty$ \\
\hline
\end{tabular}
\medskip

$\varphi\left(\Gamma_{\{\overline{(5,2)}\}}\right)=16+2=18$.

В нашем случае $\varphi\left(\Gamma_{\{\overline{(5,2)}\}}\right)=\varphi\left(\Gamma_{\{(5,2)\}}\right)=18$, значит для дальнейшей обработки мы можем выбрать любое из множеств $\Gamma_{\{(5,2)\}}$ или $\Gamma_{\{\overline{(5,2)}\}}$, пусть это будет множество $\Gamma_{\{(5,2)\}}$.

Посчитаем веса нулей в матрице $C_{1,1}$:

\medskip
\begin{tabular}{|>{\columncolor{Gray}}c|c|c|c|c|}
\hline
\rowcolor{Gray}
\cellcolor{white} & 1 & 3 & 4 & 5 \\
\hline
1 & $\infty$ & $0_{(0)}$ & $0_{(3)}$ & $0_{(1)}$ \\
\hline
2 & 2 & $0_{(2)}$ & 3 & $\infty$ \\
\hline
3 & $0_{(2)}$ & $\infty$ & 3 & 9 \\
\hline
4 & $0_{(1)}$ & 3 & $\infty$ & 1 \\
\hline
\end{tabular}
\medskip

Самым тяжёлым нулём является клетка с номером $(1,4)$, так что теперь следует рассматривать множества $\Gamma_{\{(5,2);(1,4)\}}$ и $\Gamma_{\{(5,2);\overline{(1,4)}\}}$.

Матрицу $C_{1,1,1}$ получим вычеркиванием четвёртого столбца и первой строки из матрицы $C_{1,1}$, причём число в клетке $(4,1)$ нужно будет заменить на $\infty$:

\medskip
\begin{tabular}{|>{\columncolor{Gray}}c|c|c|c|}
\hline
\rowcolor{Gray}
\cellcolor{white} & 1 & 3 & 5 \\
\hline
2 & 2 & 0 & $\infty$ \\
\hline
3 & 0 & $\infty$ & 9 \\
\hline
4 & $\infty$ & 3 & 1 \\
\hline
\end{tabular}
\medskip

Приведём матрицу по строкам:

\medskip
\begin{tabular}{|>{\columncolor{Gray}}c|c|c|c|}
\hline
\rowcolor{Gray}
\cellcolor{white} & 1 & 3 & 5 \\
\hline
2 & 2 & 0 & $\infty$ \\
\hline
3 & 0 & $\infty$ & 9 \\
\hline
4 & $\infty$ & 2 & 0 \\
\hline
\end{tabular}
\medskip

По столбцам она теперь уже стала приведённой.

$\varphi\left(\Gamma_{\{(5,2);(1,4)\}}\right)=18+1$.

Матрицу же $C_{1,1,2}$, получим заменой на $\infty$ элемента $(1,4)$ в матрице $C_{1,1}$:

\medskip
\begin{tabular}{|>{\columncolor{Gray}}c|c|c|c|c|}
\hline
\rowcolor{Gray}
\cellcolor{white} & 1 & 3 & 4 & 5 \\
\hline
1 & $\infty$ & 0 & $\infty$ & 0 \\
\hline
2 & 2 & 0 & 3 & $\infty$ \\
\hline
3 & 0 & $\infty$ & 3 & 9 \\
\hline
4 & 0 & 3 & $\infty$ & 1 \\
\hline
\end{tabular}
\medskip

По строкам она уже приведённая, а после приведения по столбцам она будет иметь вид:

\medskip
\begin{tabular}{|>{\columncolor{Gray}}c|c|c|c|c|}
\hline
\rowcolor{Gray}
\cellcolor{white} & 1 & 3 & 4 & 5 \\
\hline
1 & $\infty$ & 0 & $\infty$ & 0 \\
\hline
2 & 2 & 0 & 0 & $\infty$ \\
\hline
3 & 0 & $\infty$ & 0 & 9 \\
\hline
4 & 0 & 3 & $\infty$ & 1 \\
\hline
\end{tabular}
\medskip

$\varphi\left(\Gamma_{\{(5,2);\overline{(1,4)}\}}\right)=18+3=21$.

В этом случае сумма констант приведения больше чем в предыдущем,
следовательно дальнейшей обработке подлежит множество $\Gamma_{\{(5,2);(1,4)\}}$.


Найдём веса нулей матрицы $C_{1,1,1}$:

\medskip
\begin{tabular}{|>{\columncolor{Gray}}c|c|c|c|}
\hline
\rowcolor{Gray}
\cellcolor{white} & 1 & 3 & 5 \\
\hline
2 & 2 & $0_{(4)}$ & $\infty$ \\
\hline
3 & $0_{(11)}$ & $\infty$ & 9 \\
\hline
4 & $\infty$ & 2 & $0_{(11)}$ \\
\hline
\end{tabular}
\medskip

Самыми тяжёлыми являются нули с номерами $(4,5)$ и $(3,1)$, возьмём один из них $(3,1)$, теперь следует рассматривать множества $\Gamma_{\{(5,2);(1,4);(3,1)\}}$ и $\Gamma_{\{(5,2);(1,4);\overline{(3,1)}\}}$.

Матрицу $C_{1,1,1,1}$ получим вычеркиванием первого столбца и третьей строки из матрицы $C_{1,1,1}$, причём число в клетке $(4,3)$ нужно будет заменить на $\infty$:

\medskip
\begin{tabular}{|>{\columncolor{Gray}}c|c|c|}
\hline
\rowcolor{Gray}
\cellcolor{white} & 3 & 5 \\
\hline
2 & 0 & $\infty$ \\
\hline
4 & $\infty$ & 0 \\
\hline
\end{tabular}
\medskip

Она у нас сразу приведённая и $\varphi\left(\Gamma_{\{(5,2);(1,4);(3,1)\}}\right)=19$.

Раз сумма констант последнего приведения равна нулю, то ветвь соответствующую множеству $\Gamma_{\{(5,2);(1,4);\overline{(3,1)}\}}$ можно не рассматривать.

Нулевые клетки матрицы $C_{1,1,1,1}$ дают те дуги, которые с найденными ранее и составляют обход коммивояжёра, причём вес этого обхода равен значению оценочной функции, т.е. 19. Вот этот обход:

$(1,4) (4,5) (5,2) (2,3) (3,1)$ или
$1\to4\to5\to2\to3\to1$.

\end{document}
