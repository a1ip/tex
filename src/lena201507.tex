\documentclass{article}
%Russian-specific packages
%--------------------------------------
\usepackage[T2A]{fontenc}
\usepackage[utf8]{inputenc}
\usepackage[russian]{babel}
\usepackage{titlesec} % package to customize chapters, sections and subsections style
%--------------------------------------
\titleformat{\section}{\large\bfseries\centering}{\thesection}{1em}{}
%Hyphenation rules
%--------------------------------------
\usepackage{hyphenat}
\hyphenation{мате-мати-ка восста-навливать}
%--------------------------------------
\title{Для Лены}
\author{Ригованов Филипп Юрьевич}
\date{Июль 2015}

% Use wide margins, but not quite so wide as fullpage.sty
\marginparwidth 0.5in
\oddsidemargin 0.25in
\evensidemargin 0.25in
\marginparsep 0.25in
\topmargin 0.25in
\textwidth 6in \textheight 8in
% That's about enough definitions

\usepackage{amsmath}
\usepackage{upgreek}

\begin{document}

\maketitle
\section*{Вариант № 1.}
\begin{enumerate}

\item % Задача 1
Вычислите сумму последовательности:
$$1-\frac{3}{5}+\frac{9}{25}-\frac{27}{75}+\frac{81}{375}+\ldots$$
\item % Задача 2
Найдите производную функции:

a) $y=\frac{\cos{x}}{e^x}$

Решение:
$$y^\prime=\left(\frac{\cos{x}}{e^x}\right)^\prime=\left(\cos{x}\cdot e^{-x}\right)^\prime=\cos{x}\cdot\left(e^{-x}\right)^\prime+\left(\cos{x}\right)^\prime\cdot e^{-x}=\cos{x}\cdot\left(-e^{-x}\right)+\left(-\sin{x}\right)\cdot e^{-x}=$$ $$=-e^{-x}\left(\cos{x}+\sin{x}\right)$$

б) $y=x^{11}\cdot\sin{x}$

Решение:
$$y^\prime=\left(x^{11}\cdot\sin{x}\right)^\prime=x^{11}\cdot\left(\sin{x}\right)^\prime+\left(x^{11}\right)^\prime\cdot\sin{x}=x^{11}\cdot\cos{x}+11x^{10}\cdot\sin{x}$$

\item % Задача 3
Найдите промежутки монотонности функции:
$$y=2x^3-3x^2-36x+40$$
Решение:
$$y^\prime=\left(2x^3-3x^2-36x+4\right)^\prime=6x^2-6x-36=6\left(x^2-x-6\right)$$ % =6(x-3)(x+2)

Приравняем производную к нулю и найдём корни уравнения $x^2-x-6=0$:
$$x_1=\frac{1-\sqrt{(1)^2-4\cdot1\cdot(-6)}}{2}=\frac{1-\sqrt{25}}{2}=\frac{1-5}{2}=\frac{-4}{2}=-2$$
$$x_2=\frac{1+\sqrt{(1)^2-4\cdot1\cdot(-6)}}{2}=\frac{1+\sqrt{25}}{2}=\frac{1+5}{2}=\frac{6}{2}=3$$

\item % Задача 4
Найдите точки экстремума:
$$y=x^3-3x^2$$

\item % Задача 5
Вычислите интеграл:
$$\int{\left(5-e^x+3\cos{x}+x^4\right)dx}$$

\item % Задача 6
Найдите корни уравнения:
$$5^{x+1}+3\cdot5^{x-1}-6\cdot5^x+10=0$$

\item % Задача 7
Решите уравнение:
$$\log_2{(2x-18)}+\log_2{(x-9)}=5$$

\item % Задача 8
Решите неравенство:
$$\frac{3x-6}{2x^2+5x-3}\leq0$$

\item % Задача 9
В вазе лежат яблоки: 12 жёлтых и 6 красных. Сколькими способами можно взять из вазы 4 жёлтых и 2 красных яблока?

\item % Задача 10
В сборнике билетов по биологии всего 25 билетов, в двух из них встречается вопрос о грибах. На экзамене школьнику достаётся один случайно выбранный билет из этого сборника. Найдите вероятность того, что в этом билете не будет вопроса о грибах.

\end{enumerate}
\section*{Вариант № 2.}
\begin{enumerate}

\item % Задача 1
Вычислите сумму последовательности:
$$1-\frac{2}{3}+\frac{4}{9}-\frac{8}{27}+\frac{16}{81}-\ldots$$

\item % Задача 2
Найдите производную функции:

a) $y=\frac{3^x}{\sin{x}}$

б) $y=x^{7}\cdot\cos{x}$

\item % Задача 3
Найдите промежутки монотонности функции:
$$y=x^3-2x^2+x-2$$

\item % Задача 4
Найдите точки экстремума:
$$y=x^3+3x^2-6$$

\item % Задача 5
Вычислите интеграл:
$$\int{\left(1+3e^x-4\cos{x}+x^3\right)dx}$$

\item % Задача 6
Найдите корни уравнения:
$$2\cdot3^{x+1}-6\cdot3^{x-1}-3^x=9$$

\item % Задача 7
Решите уравнение:
$$\log_3{(x-7)}+\log_3{(x+1)}=2$$

\item % Задача 8
Решите неравенство:
$$\frac{3x-5}{x^2+5x-14}\geq0$$

\item % Задача 9
В вазе лежат яблоки: 10 зелёных и 5 красных. Сколькими способами можно взять из вазы 2 зелёных и 3 красных яблока?

\item % Задача 10
В сборнике билетов по биологии всего 25 билетов, в двух из них встречается вопрос о грибах. На экзамене школьнику достаётся один случайно выбранный билет из этого сборника. Найдите вероятность того, что в этом билете будет вопрос о грибах.

\end{enumerate}
\end{document}
