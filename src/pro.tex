\documentclass{article}
\usepackage{fullpage}
%%% Работа с русским языком
\usepackage[T2A]{fontenc}
\usepackage[utf8]{inputenc}
\usepackage[english,russian]{babel}   %% загружает пакет многоязыковой вёрстки
%\usepackage{fontspec}      %% подготавливает загрузку шрифтов Open Type, True Type и др.
%\defaultfontfeatures{Ligatures={TeX},Renderer=Basic}  %% свойства шрифтов по умолчанию
%\setmainfont[Ligatures={TeX,Historic}]{Times New Roman} %% задаёт основной шрифт документа
%\setsansfont{Comic Sans MS}                    %% задаёт шрифт без засечек
%\setmonofont{Courier New}
\usepackage{indentfirst}
\frenchspacing
\usepackage{graphicx}
\usepackage{titlesec} % package to customize chapters, sections and subsections style
%--------------------------------------
\titleformat{\section}{\large\bfseries\centering}{\thesection}{1em}{}
%Hyphenation rules
%--------------------------------------
\usepackage{hyphenat}
\hyphenation{мате-мати-ка восста-навливать}
%--------------------------------------
%%% Дополнительная работа с математикой
\usepackage[intlimits,sumlimits]{amsmath}
\usepackage{amsfonts,amssymb,amsthm,mathtools} % AMS
%\usepackage{amsmath,amsfonts,amssymb,amsthm,mathtools} % AMS
\usepackage{icomma} % "Умная" запятая: $0,2$ --- число, $0, 2$ --- перечисление

%% Перенос знаков в формулах (по Львовскому)
\newcommand*{\hm}[1]{#1\nobreak\discretionary{}
{\hbox{$\mathsurround=0pt #1$}}{}}

\renewcommand{\le}{\ensuremath{\leqslant}}
\renewcommand{\leq}{\ensuremath{\leqslant}}
\renewcommand{\ge}{\ensuremath{\geqslant}}
\renewcommand{\geq}{\ensuremath{\geqslant}}

\title{Контрольная работа}
\author{Теория вероятностей и математическая статистика}
\date{Декабрь 2015}

\begin{document}

\maketitle
\section*{Вариант № 7.}
Задание 4

Найти вероятность того, что в $n$ независимых испытаний событие появится не менее $k$ раз, зная, что в каждом испытании вероятность появления события равна $p$.

$$n=5; k=3; p=0,7.$$
\begin{center}Решение:\end{center}
Используем формулу Бернулли $P_n(k)=C_n^k p^k q^{n-k}$, вероятность того, что событие наступит не менее $k$ раз с помощью неё вычисляется так: $$P_n(i\ge k)=\sum_{i=k}^n P_n(i)$$
В нашем случае $n=5; k=3; p=0,7; q=0,3;$ значит
$$P_5(i\ge 3)=\sum_{i=3}^5 P_5(i)=P_5(3)+P_5(4)+P_5(5)=C_5^3\cdot0,7^3\cdot0,3^2+C_5^4\cdot0,7^4\cdot0,3+C_5^5\cdot0,7^5=$$
$$=10\cdot0,343\cdot0,09+5\cdot0,2401\cdot0,3+0,16807=0,83692.$$

Ответ: $0,83692$.

\end{document}
