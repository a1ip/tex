\documentclass[fleqn]{article}
\usepackage{fullpage}
%%% Работа с русским языком
\usepackage[T2A]{fontenc}
\usepackage[utf8]{inputenc}
\usepackage[english,russian]{babel}   %% загружает пакет многоязыковой вёрстки
\usepackage{indentfirst}
\frenchspacing
\usepackage{titlesec} % package to customize chapters, sections and subsections style
%--------------------------------------
\titleformat{\section}{\large\bfseries\centering}{\thesection}{1em}{}
%Hyphenation rules
%--------------------------------------
\usepackage{hyphenat}
\hyphenation{мате-мати-ка восста-навливать}
%--------------------------------------
\usepackage[intlimits,sumlimits,fleqn]{amsmath}
\usepackage{amsfonts,amssymb,amsthm,mathtools}
\usepackage[bb=pazo,frak=pxtx,frakscaled=1.3]{mathalfa}
\usepackage{upgreek}
\usepackage{graphicx}
\usepackage{verbatim} %% для многострочных комментариев

%% Перенос знаков в формулах (по Львовскому)
\newcommand*{\hm}[1]{#1\nobreak\discretionary{}
{\hbox{$\mathsurround=0pt #1$}}{}}

\renewcommand{\le}{\ensuremath{\leqslant}}
\renewcommand{\leq}{\ensuremath{\leqslant}}
\renewcommand{\ge}{\ensuremath{\geqslant}}
\renewcommand{\geq}{\ensuremath{\geqslant}}

\newcommand{\tg}{\mathop{\rm tg}\nolimits}

\usepackage{tikz}
\usepackage{pgfplots}
\pgfplotsset{width=10cm,compat=1.9}
\usepgfplotslibrary{external}
\tikzexternalize
\tikzset{>=latex}

\usetikzlibrary{arrows.meta}

\usepackage{xcolor,colortbl}

\newcommand{\mc}[2]{\multicolumn{#1}{c}{#2}}
\definecolor{Gray}{gray}{0.85}

\newcolumntype{a}{>{\columncolor{Gray}}c}
\newcolumntype{b}{>{\columncolor{white}}c}

\DeclareMathOperator{\sign}{sign}

\title{Контрольная работа по курсу «Элементы дискретной математики»\\
Вариант №2}

\begin{document}
\date{}
\maketitle
\section*{Задание № 1.}

Даны множества чисел: $A=\{0,1,3,4\}$, $B=\{3,4,5,6\}$, $C=\{1,2,4,6\}$ и универсальное множество $U=\{0,1,2,3,4,5,6,7\}$.

Найти множества чисел $D=\overline{A\cup B}\cup (C\cap B),\;\; E=(\overline{B}\cap \overline{A})\cup (C\cap(B\setminus A))$. Являются ли множества $E$ и $D$ равными? эквивалентными? включающими одно в другое ( $D \subset E$ или $E \subset D$ )? пересекающимися, но не включающими одно в другое? непересекающимися $( D \cap E = \varnothing )$?

\begin{center}Решение:\end{center}

Для нахождения множества $D$ сначала найдем: объединение $A\cup B=\{0,1,3,4,5,6\}$, и его дополнение (до множества $U$) $\overline{A\cup B}=\{2,7\}$, затем пересечение $C\cap B=\{4,6\}$. Теперь $D=\{2,7\}\cup\{4,6\}=\{2,4,6,7\}$.

Для нахождения множества $E$ сначала найдем: дополнения множеств $\overline{B}$ и $\overline{A}$ (до множества $U$) $\overline{B}=\{0,1,2,7\}$, $\overline{A}=\{2,5,6,7\}$ и их пересечение $\overline{B}\cap\overline{A}=\{2,7\}$, а так же разность множеств $B\setminus A=\{5,6\}$ и пересечение $C\cap(B\setminus A)=\{6\}$. Теперь $E=\{2,7\}\cup\{6\}=\{2,6,7\}$.

Множества $D$ и $E$ не равные потому что $4\in D$, но $4\notin E$, и не эквивалентные, так как имеют разные мощности (4 и 3 соответсвенно), причём множество $E$ включается в множество $D$ ($E \subset D$).

\section*{Задание № 2.}

Из $100$ работников фирмы $42$ владеют английским языком, $30$ – французским, $28$ – немецким. Десять человек – знают английский и немецкий, $8$ – знают французский и немецкий, $5$ – английский и французский. Три человека знают все три языка. Сколько работников фирмы не знают ни одного языка?

Решить задачу, используя теорию множеств.


\begin{center}Решение:\end{center}

Пусть $|A|$, $|B|$ и $|C|$ --- число работников фирмы, владеющих аглийским, французским и немецким языками соответственно. По условию $|A|=42$, $|B|=30$, $|C|=28$, $|A\cap B|=5$, $|A\cap C|=10$, $|B\cap C|=8$, $|A\cap B\cap C|=3$. Тогда по формуле включений-выключений число работников владеющих хотя бы одним из этих языков:

$$|А\cup B\cup C|=|A|+|B|+|C|-|A\cap B|-|A\cap C|-|B\cap C|+|A\cap B\cap C|=42+30+28-5-10-8+3=80.$$

Следовательно число работников не знающих ни одного из этих языков равно $100-80=20$.

\section*{Задание № 3.}

Установить вид формулы алгебры логики:
$$L = ((A \wedge B) \vee C) \leftrightarrow (A \to (B \vee \overline{C} )).$$

\begin{center}Решение:\end{center}

\medskip
\begin{tabular}{|c|c|c|c|c|c|c|c|}
\hline
$A$ & $B$ & $C$ & $A \wedge B$ & $(A \wedge B) \vee C$ & $B \vee \overline{C}$ & $A\to(B\vee\overline{C})$ & $L$ \\
\hline
0 & 0 & 0 & 0 & 0 & 1 & 1 & 0 \\
\hline
0 & 0 & 1 & 0 & 1 & 0 & 1 & 1 \\
\hline
0 & 1 & 0 & 0 & 0 & 1 & 1 & 0 \\
\hline
0 & 1 & 1 & 0 & 1 & 1 & 1 & 1 \\
\hline
1 & 0 & 0 & 0 & 0 & 1 & 1 & 0 \\
\hline
1 & 0 & 1 & 0 & 1 & 0 & 0 & 0 \\
\hline
1 & 1 & 0 & 1 & 1 & 1 & 1 & 1 \\
\hline
1 & 1 & 1 & 1 & 1 & 1 & 1 & 1 \\
\hline
\end{tabular}
\medskip

Из полученной таблицы видно, что формула $L$ является выполнимой, так как она принимает значение $1$, но не является тождественно выполнимой (тавтологией), ибо при определенных значениях высказываний она принимает значение $0$.

\section*{Задание № 4.}

С помощью таблицы истинности найти СДНФ и СКНФ булевой
функции $f(x_1,x_2)=x_1 \wedge x_2 \to(x_1 \to \overline{x_2})$.

\begin{center}Решение:\end{center}

\medskip
\begin{tabular}{|c|c|c|c|c|c|}
\hline
$x_1$ & $x_2$ & $\overline{x_2}$ & $x_1 \wedge x_2$ & $x_1 \to \overline{x_2}$ & $f(x_1,x_2)$ \\
\hline
0 & 0 & 1 & 0 & 1 & 1 \\
\hline
0 & 1 & 0 & 0 & 1 & 1 \\
\hline
1 & 0 & 1 & 0 & 1 & 1 \\
\hline
1 & 1 & 0 & 1 & 0 & 0 \\
\hline
\end{tabular}
\medskip

1) Функция $f(x_1, x_2)$ равна $1$ на наборах $(x_1, x_2)$: $(0;0)$, $(0;1)$ и $(1;0)$, т.е. соответствующие конъюнкции (над равными $0$ переменными ставим знак отрицания) $\overline{x_1} \wedge \overline{x_2}$, $\overline{x_1} \wedge x_2$ и $x_1 \wedge \overline{x_2}$. Соединяя их знаками дизъюнкции, получим СДНФ функции:

$$f(x_1, x_2)=(\overline{x_1}\wedge\overline{x_2})\vee(\overline{x_1} \wedge x_2)\vee(x_1\wedge\overline{x_2}).$$

2) Функция $f(x_1, x_2)$ равна $0$ только на наборе $(x_1, x_2)$: $(1;1)$, т.е. соответствующая дизъюнкция (над равными $1$ переменными ставим знак отрицания) $\overline{x_1} \vee \overline{x_2}$. Это и будет СКНФ функции:

$$f(x_1, x_2)=\overline{x_1} \vee \overline{x_2}.$$

\section*{Задание № 5.}

Дана матрица:

$$A=
\begin{pmatrix}
0 & 1 & 1 & 0 & 1 \\
0 & 0 & 1 & 0 & 0 \\
1 & 0 & 0 & 0 & 0 \\
1 & 0 & 1 & 0 & 0 \\
0 & 0 & 0 & 1 & 0 \\
\end{pmatrix}.
$$

Построить ориентированный граф, для которого матрица A является матрицей смежности. Найти матрицу инцидентности.

Является ли полученный граф связным?

\begin{center}Решение:\end{center}

\medskip
\begin{tikzpicture}
\begin{scope}[every node/.style={circle,thick,draw}]
    \node (A) at (-3,3) {$v_1$};
    \node (B) at (3,3) {$v_2$};
    \node (C) at (4,-2) {$v_3$};
    \node (D) at (0,-4) {$v_4$};
    \node (E) at (-4,-2) {$v_5$};
\end{scope}

\begin{scope}[>={Stealth[black]},
              every node/.style={fill=white,circle},
              every edge/.style={draw=red,very thick}]
    \path [->] (A) edge node {$x_1$} (B);
    \path [->] (A) edge[bend left=12] node {$x_2$} (C);
    \path [->] (A) edge node {$x_3$} (E);
    \path [->] (B) edge node {$x_4$} (C);
    \path [->] (C) edge[bend left=12] node {$x_5$} (A);
    \path [->] (D) edge node {$x_6$} (A);
    \path [->] (D) edge node {$x_7$} (C);
    \path [->] (E) edge node {$x_8$} (D);
\end{scope}
\end{tikzpicture}
\medskip

Построим матрицу инцидентности:

$$B=
\begin{pmatrix}
1 & 1 & 1 & 0 & -1 & -1 & 0 & 0 \\
-1 & 0 & 0 & 1 & 0 & 0 & 0 & 0 \\
0 & -1 & 0 & -1 & 1 & 0 & -1 & 0 \\
0 & 0 & 0 & 0 & 0 & 1 & 1 & -1 \\
0 & 0 & -1 & 0 & 0 & 0 & 0 & 1 \\
\end{pmatrix}
$$

Данный граф является связным потому что из любой его вершины можно попасть в любую, например по циклу $x_1\to x_4 \to x_2 \to x_3 \to x_8 \to x_6$.

\section*{Задание № 6.}

Определить функцию $f(x,y)$, полученную из функций $g(x) =1$ и
$h(x, y, z) = x+z$ по схеме примитивной рекурсии.

\begin{center}Решение:\end{center}

Имеем по определению оператора примитивной рекурсии:

1) $f(x,0)=g(x)=1$;

2) $f(x,y+1)=h\left(x,y,f(x,y)\right)=x+f(x,y)$.

Отсюда имеем:

$f(x,1)=x+f(x,0)=x+1$;

$f(x,2)=x+f(x,1)=x+(x+1)=2x+1$;

$f(x,3)=x+f(x,2)=x+(2x+1)=3x+1$ и так далее.

Можно предположить, что $f(x,y)=yx+1$, и доказать эту формулу
методом математической индукции по переменной $y$.

\end{document}
