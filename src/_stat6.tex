\documentclass[fleqn]{article}
\usepackage{fullpage}
%%% Работа с русским языком
\usepackage[T2A]{fontenc}
\usepackage[utf8]{inputenc}
\usepackage[english,russian]{babel}   %% загружает пакет многоязыковой вёрстки
\usepackage{indentfirst}
\frenchspacing
\usepackage{titlesec} % package to customize chapters, sections and subsections style
%--------------------------------------
\titleformat{\section}{\large\bfseries\centering}{\thesection}{1em}{}
%Hyphenation rules
%--------------------------------------
\usepackage{hyphenat}
\hyphenation{мате-мати-ка восста-навливать}
%--------------------------------------
\usepackage[intlimits,sumlimits,fleqn]{amsmath}
\usepackage{amsfonts,amssymb,amsthm,mathtools}
\usepackage[bb=pazo,frak=pxtx,frakscaled=1.3]{mathalfa}
\usepackage{upgreek}
\usepackage{graphicx}
\usepackage{verbatim} %% для многострочных комментариев

%% Перенос знаков в формулах (по Львовскому)
\newcommand*{\hm}[1]{#1\nobreak\discretionary{}
{\hbox{$\mathsurround=0pt #1$}}{}}

\renewcommand{\le}{\ensuremath{\leqslant}}
\renewcommand{\leq}{\ensuremath{\leqslant}}
\renewcommand{\ge}{\ensuremath{\geqslant}}
\renewcommand{\geq}{\ensuremath{\geqslant}}

\newcommand{\tg}{\mathop{\rm tg}\nolimits}

\usepackage{tikz}
\usepackage{pgfplots}
\pgfplotsset{width=10cm,compat=1.9}
\usepgfplotslibrary{external}
\tikzexternalize
\tikzset{>=latex}

\usetikzlibrary{arrows.meta}

\usepackage{multirow}

\usepackage{xcolor,colortbl}

\newcommand{\mc}[2]{\multicolumn{#1}{c}{#2}}
\definecolor{Gray}{gray}{0.85}

\newcolumntype{a}{>{\columncolor{Gray}}c}
\newcolumntype{b}{>{\columncolor{white}}c}

\DeclareMathOperator{\sign}{sign}

\title{Контрольная работа по дициплине «Статистика»\\
Вариант №6}

\begin{document}
\date{}
\maketitle
\section*{Вариационные ряды распределения}

На основе уловных ранжированных данных таблицы провести анализ вариации величины налоговых сборов (тыс. руб.) с предприятий района, собранных налоговыми органами.

1. Построить интервальный ряд распределения.

2. Рассчитать медиану.

3. Рассчитать моду.

\bgroup
\def\arraystretch{1.5}
\setlength\tabcolsep{4}
\begin{center}
\begin{tabular}{|c|c|c|c|c|c|c|c|c|c|c|c|c|c|c|c|c|c|c|c|c|}
\hline
$i$ & 1 & 2 & 3 & 4 & 5 & 6 & 7 & 8 & 9 & 10 & 11 & 12 & 13 & 14 & 15 & 16 & 17 & 18 & 19 & 20 \\
\hline
$x_i$ & 101 & 163 & 200 & 230 & 308 & 314 & 320 & 328 & 367 & 405 & 410 & 420 & 427 & 440 & 458 & 464 & 465 & 482 & 495 & 497 \\
\hline
\end{tabular}
\end{center}

\begin{center}
\begin{tabular}{|c|c|c|c|c|c|c|c|c|c|c|c|c|c|c|c|c|c|c|c|c|}
\hline
$i$ & 21 & 22 & 23 & 24 & 25 & 26 & 27 & 28 & 29 & 30 & 31 & 32 & 33 & 34 & 35 & 36 & 37 & 38 & 39 & 40 \\
\hline
$x_i$ & 545 & 549 & 582 & 590 & 602 & 603 & 614 & 621 & 633 & 643 & 664 & 666 & 676 & 691 & 692 & 708 & 717 & 726 & 743 & 744 \\
\hline
\end{tabular}
\end{center}

\begin{center}
\begin{tabular}{|c|c|c|c|c|c|c|c|c|c|c|}
\hline
$i$ & 41 & 42 & 43 & 44 & 45 & 46 & 47 & 48 & 49 & 50 \\
\hline
$x_i$ & 753 & 758 & 772 & 793 & 808 & 861 & 867 & 880 & 897 & 929 \\
\hline
\end{tabular}
\end{center}
\egroup

\begin{center}Решение:\end{center}

Найдём число групп в ряду распределения по формуле Стерджесса:

$$k=1+\log_2{50}\approx7.$$

Тогда длина интервала (округлим до десятков) будет:

$$h=\frac{929-101}{7}\approx118\approx120\textit{(тыс. руб.)}.$$

Построим таблицу распределения частот по интервалам.

\bgroup
\def\arraystretch{1.5}
\setlength\tabcolsep{6}

\begin{center}
\begin{tabular}{|c|c|c|c|c|c|c|c|}
\hline
Интервалы, тыс. руб. & 100-220 & 220-340 & 340-460 & 460-580 & 580-700 & 700-820 & 820-940 \\
\hline
$f_i$(частота) & 3 & 5 & 7 & 7 & 13 & 10 & 5 \\
\hline
$w_i,\%$(частость) & 6 & 10 & 14 & 14 & 26 & 20 & 10 \\
\hline
$F_i$(накопленная частота) & 3 & 8 & 15 & 22 & 35 & 45 & 50 \\
\hline
$P_i,\%$(накопленная частость) & 6 & 16 & 30 & 44 & 70 & 90 & 100 \\
\hline
\end{tabular}
\end{center}
\egroup

Из таблицы распределения частот по интервалам видно, что медианным интервалом является интервал $580-700$ поскольку именно на нём накопленная частость $P_i$ переходит за $50\%$. Вычислим значение медианы:

$$Me=580+120\cdot\frac{\frac{50}{2}-22}{13}=580+\frac{360}{13}\approx608\textit{(тыс. руб.)}.$$

Этот же интервал является и модальным, поскольку именно на него приходится наибольшая частота. Рассчитаем величину моды:

$$Mo=580+120\cdot\frac{13-7}{(13-7)+(13-10)}=580+120\cdot\frac{6}{6+3}=580+\frac{240}{3}=580+80=660\textit{(тыс. руб.)}.$$

\bigskip

\section*{Анализ рядов динамики}

Проанализировать ряд динамики, приведённый в таблице.

\newcommand{\specialcell}[2][c]{%
  \begin{tabular}[#1]{@{}c@{}}#2\end{tabular}}

\bgroup
\def\arraystretch{1.5}
\setlength\tabcolsep{7}
\begin{center}
\begin{tabular}{|c|c|c|c|c|c|c|c|}
\hline
Год & 2007 & 2008 & 2009 & 2010 & 2011 & 2012 & 2013 \\
\hline
\specialcell{Численность лиц,\\впервые признанных\\инвалидами, тыс. чел.} & 1109 & 1200 & 1184 & 1092 & 1463 & 1799 & 1443 \\
\hline
\end{tabular}
\end{center}
\egroup

\begin{center}Решение:\end{center}

Вычислим базисные абсолютные изменения:

$$\Delta y_2^\textit{Б}=y_2-y_1=1200-1109=91;\;\Delta y_3^\textit{Б}=y_3-y_1=1184-1109=75;\;\Delta y_4^\textit{Б}=y_4-y_1=1092-1109=-17;$$
$$\Delta y_5^\textit{Б}=y_5-y_1=1463-1109=354;\;\Delta y_6^\textit{Б}=y_6-y_1=1799-1109=690;\;\Delta y_7^\textit{Б}=y_7-y_1=1443-1109=334.$$

Теперь цепные абсолютные изменения:

$$\Delta y_2^\textit{Ц}=y_2-y_1=1200-1109=91;\;\Delta y_3^\textit{Ц}=y_3-y_2=1184-1200=-16;\;\Delta y_4^\textit{Ц}=y_4-y_3=1092-1184=-92;$$
$$\Delta y_5^\textit{Ц}=y_5-y_4=1463-1092=371;\;\Delta y_6^\textit{Ц}=y_6-y_5=1799-1463=336;\;\Delta y_7^\textit{Ц}=y_7-y_6=1443-1799=-356.$$

Правильность расчётов подтверждается тем что:

$$\sum_{i=1}^7\Delta y_i^\textit{Ц}=91-16-92+371+336-356=334=\Delta y_7^\textit{Б}.$$

Далее вычислим базовые относительные изменения:

$$i_2^\textit{Б}=y_2/y_1=1200/1109\approx1,082;\;i_3^\textit{Б}=y_3/y_1=1184/1109\approx1,068;\;i_4^\textit{Б}=y_4/y_1=1092/1109\approx0,985;$$
$$i_5^\textit{Б}=y_5/y_1=1463/1109\approx1,319;\;i_6^\textit{Б}=y_6/y_1=1799/1109\approx1,622;\;i_7^\textit{Б}=y_7/y_1=1443/1109\approx1,3.$$

А теперь цепные относительные изменения:

$$i_2^\textit{Ц}=y_2/y_1=1200/1109\approx1,082;\;i_3^\textit{Ц}=y_3/y_2=1184/1200\approx0,987;\;i_4^\textit{Ц}=y_4/y_3=1092/1184\approx0,922;$$
$$i_5^\textit{Ц}=y_5/y_4=1463/1092\approx1,34;\;i_6^\textit{Ц}=y_6/y_5=1799/1463\approx1,23;\;i_7^\textit{Ц}=y_7/y_6=1443/1799\approx0,802.$$

Правильность расчётов подтверждается тем что:

$$\prod_{i=1}^7i_i^\textit{Ц}=1,082\cdot0,987\cdot0,922\cdot1,34\cdot1,23\cdot0,802\approx1,3\approx i_7^\textit{Б}.$$

Вычислим базисные темпы изменения:

$$T_2^\textit{Б}=(i_2^\textit{Б}-1)\cdot100\approx(1,082-1)\cdot100\approx8,2;\;T_3^\textit{Б}=(i_3^\textit{Б}-1)\cdot100\approx(1,068-1)\cdot100\approx6,8;$$
$$T_4^\textit{Б}=(i_4^\textit{Б}-1)\cdot100\approx(0,985-1)\cdot100\approx-1,5;\;T_5^\textit{Б}=(i_5^\textit{Б}-1)\cdot100\approx(1,319-1)\cdot100\approx31,9;$$
$$T_6^\textit{Б}=(i_6^\textit{Б}-1)\cdot100\approx(1,622-1)\cdot100\approx62,2;\;T_7^\textit{Б}=(i_7^\textit{Б}-1)\cdot100\approx(1,3-1)\cdot100\approx30.$$

И наконец цепные темпы изменения:

$$T_2^\textit{Ц}=(i_2^\textit{Ц}-1)\cdot100\approx(1,082-1)\cdot100\approx8,2;\;T_3^\textit{Ц}=(i_3^\textit{Ц}-1)\cdot100\approx(0,987-1)\cdot100\approx-1,3;$$
$$T_4^\textit{Ц}=(i_4^\textit{Ц}-1)\cdot100\approx(0,922-1)\cdot100\approx-7,8;\;T_5^\textit{Ц}=(i_5^\textit{Ц}-1)\cdot100\approx(1,34-1)\cdot100\approx34;$$
$$T_6^\textit{Ц}=(i_6^\textit{Ц}-1)\cdot100\approx(1,23-1)\cdot100\approx23;\;T_7^\textit{Ц}=(i_7^\textit{Ц}-1)\cdot100\approx(0,802-1)\cdot100\approx-19,8.$$

В последней строке запишем суммы элементов соответствующих столбцов и требуемая таблица готова.

\bgroup
\def\arraystretch{1.8}
\setlength\tabcolsep{10}
\begin{center}
\begin{tabular}{|c|c|c|c|c|c|c|c|}
\hline
Год & $y_i$ & $\Delta y_i^\textit{Б}$ & $\Delta y_i^\textit{Ц}$ & $i_i^\textit{Б}$ & $i_i^\textit{Ц}$ & $T_i^\textit{Б}$,\% & $T_i^\textit{Ц}$,\% \\
\hline
2007 & 1109 & -- & -- & -- & -- & -- & -- \\
\hline
2008 & 1200 & 91 & 91 & $1,082$ & $1,082$ & $8,2$ & $8,2$ \\
\hline
2009 & 1184 & 75 & -16 & $1,068$ & $0,987$ & $6,8$ & $-1,3$ \\
\hline
2010 & 1092 & -17 & -92 & $0,985$ & $0,922$ & $-1,5$ & $-7,8$ \\
\hline
2011 & 1463 & 354 & 371 & $1,319$ & $1,34$ & $31,9$ & $34$ \\
\hline
2012 & 1799 & 690 & 336 & $1,622$ & $1,23$ & $62,2$ & $23$ \\
\hline
2013 & 1443 & 334 & -356 & $1,3$ & $0,802$ & $30$ & $-19,8$ \\
\hline
Итого & 9290 & 1527 & 334 & $7,376$ & $6,363$ & $137,6$ & $36,3$ \\
\hline
\end{tabular}
\end{center}
\egroup

\bigskip

\section*{Статистические индексы}

Имеются данные о продажах минимаркетом 3-х видов однородных товаров (A, B и C).

\bgroup
\def\arraystretch{1.5}
\setlength\tabcolsep{10}
\begin{center}
\begin{tabular}{|c|c|c|c|c|}
\hline
\multirow{2}{*}{\specialcell{Вид\\товара}} & \multicolumn{2}{c|}{\specialcell{Цена за единицу\\товара, руб.}} & \multicolumn{2}{c|}{\specialcell{Объём продаж,\\тыс. штук}} \\
\cline{2-5}
 & 1 квартал & 2 квартал & 1 квартал & 2 квартал \\
\hline
A & 130 & 125 & 138 & 198 \\
\hline
B & 50 & 56 & 339 & 264 \\
\hline
C & 20 & 21 & 613 & 511 \\
\hline
\end{tabular}
\end{center}
\egroup

Рассчитать индивидуальные, общие и средние индексы, выполнить факторный анализ общей выручки от продажи товаров. По итогам расчетов сделать аргументированные выводы.

\begin{center}Решение:\end{center}

Найдём индивидуальные индексы цены $i_p=\frac{p_1}{p_0}$ для каждого из товаров A, B, C:

$$i_p^A=\frac{125}{130}\approx0,96;\quad i_p^B=\frac{56}{50}=1,12;\quad i_p^C=\frac{21}{20}=1,05.$$

Индивидуальные индексы количества $i_q=\frac{q_1}{q_0}$ для каждого из товаров A, B, C:

$$i_q^A=\frac{198}{138}\approx1,43;\quad i_q^B=\frac{264}{339}\approx0,78;\quad i_q^C=\frac{511}{613}\approx0,83.$$

Индивидуальные индексы выручки $i_Q=i_q\cdot i_p$ для товаров A, B, C:

$$i_Q^A=1,43\cdot0,96\approx1,37;\quad i_Q^B=0,78\cdot1,12\approx0,87;\quad i_Q^C=0,83\cdot1,05\approx0,87.$$

Найдём общие суммы выручки в каждом из кварталов:

$$Q_0=\sum{q_0p_0}=138\cdot130+339\cdot50+613\cdot20=17940+16950+12260=47150 (\textit{тыс. руб.});$$
$$Q_1=\sum{q_1p_1}=198\cdot125+264\cdot56+511\cdot21=24750+14784+10731=50265  (\textit{тыс. руб.}).$$

Индекс общего объёма товарооборота (выручки):

$$I_Q=\frac{Q_1}{Q_0}=\frac{\sum{q_1p_1}}{\sum{q_0p_0}}=\frac{50265}{47150}\approx1,07.$$

Раз в задаче речь идёт о продаже товара в минимаркете, то цена является первым фактором, а количество вторым.

Тогда агрегатный индекс Ласпейреса:

$$I_p^\textit{Л}=\frac{\sum{p_1q_0}}{\sum{p_0q_0}}=\frac{125\cdot138+56\cdot339+21\cdot613}{47150}=\frac{17250+18984+12873}{47150}=\frac{49107}{47150}\approx1,0415.$$

А агрегатный индекс Пааше будет:

$$I_q^\textit{П}=\frac{\sum{p_1q_1}}{\sum{p_1q_0}}=\frac{50265}{49107}\approx1,0236.$$

Раз товары однородные, то можем посчитать средние цены в первом и втором кварталах:

$$\overline{p_0}=\frac{\sum{p_0q_0}}{\sum{q_0}}=\frac{47150}{138+339+613}=43,26 (\textit{руб.});\quad\overline{p_1}=\frac{\sum{p_1q_1}}{\sum{q_1}}=\frac{50265}{198+264+511}\approx51,66 (\textit{руб.}).$$

И индекс средних цен:

$$I_{\overline{p}}=\frac{\overline{p_1}}{\overline{p_0}}=\frac{51,66}{43,26}\approx1,2.$$

Проведём теперь факторный анализ:

$$\Delta Q=Q_1-Q_0=50265-47150=3115 (\textit{тыс. руб.}).$$

Товарооборот увеличился. На изменение товарооборота влияли два фактора: изменение цен и изменение количества товаров (причём цена у нас является первым фактором, а количество вторым).

$$\Delta Q=\Delta Q_p+\Delta Q_q$$

Применяя метод Чалиева вычислим:

$$\Delta Q_p=I_q^\textit{П}\cdot\left(I_p^\textit{Л}-1\right)\cdot Q_0=1,0236\cdot(1,0415-1)\cdot47150\approx2002,9 (\textit{тыс. руб.})$$

$$\Delta Q_q=\left(I_q^\textit{П}-1\right)\cdot Q_0=(1,0236-1)\cdot47150\approx1112,7 (\textit{тыс. руб.})$$

Проверка правильности расчета влияния факторов: $\Delta Q =\Delta Q_p+\Delta Q_q =2002,9+1112,7= 3115,6$ тыс. руб., что практически совпадает с общим изменением общей выручки, рассчитанным ранее, погрешность появилась при округлении.


Факторный анализ показал, что те изменения цен, которые произошли во втором квартале, должны были увеличить выручку при прежнем количестве продаж на 2002,9 тыс. руб. Однако выручка увеличилась аж на 3115 тыс. руб., значит это произошло за счёт положительного влияния второго фактора -- изменения количества.

\begin{comment}
$$\Delta Q_p=\sum{q_1(p_1-p_0)}=\sum{q_1p_1}-\sum{q_1p_0}=50265-(198\cdot130+264\cdot50+511\cdot20)=50265-49160=1105$$

$$\Delta Q_q=\sum{p_0(q_1-q_0)}=\sum{p_0q_1}-\sum{p_0q_0}=49160-47150=2010$$
\end{comment}
\end{document}
