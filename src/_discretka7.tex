\documentclass[fleqn]{article}
\usepackage{fullpage}
%%% Работа с русским языком
\usepackage[T2A]{fontenc}
\usepackage[utf8]{inputenc}
\usepackage[english,russian]{babel}   %% загружает пакет многоязыковой вёрстки
\usepackage{indentfirst}
\frenchspacing
\usepackage{titlesec} % package to customize chapters, sections and subsections style
%--------------------------------------
\titleformat{\section}{\large\bfseries\centering}{\thesection}{1em}{}
%Hyphenation rules
%--------------------------------------
\usepackage{hyphenat}
\hyphenation{мате-мати-ка восста-навливать}
%--------------------------------------
\usepackage[intlimits,sumlimits,fleqn]{amsmath}
\usepackage{amsfonts,amssymb,amsthm,mathtools}
\usepackage[bb=pazo,frak=pxtx,frakscaled=1.3]{mathalfa}
\usepackage{upgreek}
\usepackage{graphicx}
\usepackage{verbatim} %% для многострочных комментариев

%% Перенос знаков в формулах (по Львовскому)
\newcommand*{\hm}[1]{#1\nobreak\discretionary{}
{\hbox{$\mathsurround=0pt #1$}}{}}

\renewcommand{\le}{\ensuremath{\leqslant}}
\renewcommand{\leq}{\ensuremath{\leqslant}}
\renewcommand{\ge}{\ensuremath{\geqslant}}
\renewcommand{\geq}{\ensuremath{\geqslant}}

\newcommand{\tg}{\mathop{\rm tg}\nolimits}

\usepackage{tikz}
\usepackage{pgfplots}
\pgfplotsset{width=10cm,compat=1.9}
\usepgfplotslibrary{external}
\tikzexternalize
\tikzset{>=latex}

\usetikzlibrary{arrows.meta}

\usepackage{xcolor,colortbl}

\newcommand{\mc}[2]{\multicolumn{#1}{c}{#2}}
\definecolor{Gray}{gray}{0.85}

\newcolumntype{a}{>{\columncolor{Gray}}c}
\newcolumntype{b}{>{\columncolor{white}}c}

\DeclareMathOperator{\sign}{sign}

\title{Контрольная работа по курсу «Элементы дискретной математики»\\
Вариант №7}

\begin{document}
\date{}
\maketitle
\section*{Задание № 1.}

Даны множества чисел: $A=\{1,2,4,5\}$, $B=\{4,5,6,7\}$, $C=\{2,3,5,7\}$ и универсальное множество $U=\{1,2,3,4,5,6,7,8\}$.

Найти множества чисел $D=(A\cap C)\cup \overline{C\cup B},\;\; E=(\overline{B}\cap \overline{C})\cup(A\cap(C\setminus B))$. Являются ли множества $E$ и $D$ равными? эквивалентными? включающими одно в другое ( $D \subset E$ или $E \subset D$ )? пересекающимися, но не включающими одно в другое? непересекающимися $(D\cap E=\varnothing)$?

\begin{center}Решение:\end{center}

Для нахождения множества $D$ сначала найдем: пересечение $A\cap C=\{2,5\}$, объединение $C\cup B=\{2,3,4,5,6,7\}$, и его дополнение (до множества $U$) $\overline{C\cup B}=\{1,8\}$. Теперь $D=\{2,5\}\cup\{1,8\}=\{1,2,5,8\}$.

Для нахождения множества $E$ сначала найдем: дополнения множеств $B$ и $C$ (до множества $U$) $\overline{B}=\{1,2,3,8\}$ и $\overline{C}=\{1,4,6,8\}$, затем пересечение $\overline{B}\cap \overline{C}=\{1,8\}$, разность множеств $C\setminus B=\{2,3\}$, затем пересечение $A\cap(C\setminus B)=\{2\}$. Теперь $E=\{1,8\}\cup\{2\}=\{1,2,8\}$.

Множества $D$ и $E$ не равные потому что $5\in D$, но $5\notin E$, и не эквивалентные, так как имеют разные мощности (4 и 3 соответсвенно), причём множество $E$ включается в множество $D$ ($E \subset D$).

\section*{Задание № 2.}

Чемпионат по футболу проводится по круговой системе. За победу в матче даётся два очка, за ничью –- одно, а за поражение нуль. Если две команды набирают одинаковое количество очков, то место определяется по разности забитых и пропущенных мячей. Чемпион набрал семь очков, второй призер -- пять, третий –- три. Сколько очков набрала команда, занявшая последнее место?

Задачу решить, используя теорию графов.

\begin{center}Решение:\end{center}

Пусть в чемпионате участвовало $n$ команд. Построим граф $G$ встреч. Поскольку в круговой системе каждый должен был сыграть с каждым, то граф $G$ --- полный. Он имеет $\frac{n(n-1)}{2}$ рёбер. Вне зависимости от исхода матча в любом матче разыгрывается два очка. Значит за время всего чемпионата было разыграно $n(n-1)$ очков. На долю призёров приходится $15$ очков, на долю всех остальных участников $n(n-1)-15$ очков. Каждая из остальных $(n-3)$ команд, не ставших призёрами не может набрать более трёх очков (ведь ровно столько у третьего призёра). Поэтому

$$n(n-1)-15\leq3(n-3),$$

$$(n-2)^2\leq10,$$

$$n\leq5.$$

Если бы в чемпионате участвовало $3$ или $4$ команды, то они набрали бы в сумме меньше $15$ очков. Поэтому в чемпионате участвовало $5$ команд, которые в сумме набрали $5\cdot(5-1)=20$ очков. На долю двух последних приходится $5$ очков. Одна из них набрала $3$, другая $2$ очка. (При другом делении очков, например $4$ и $1$, изменится состав призёров.)

Следовательно, команда, занявшая последнее место набрала $2$ очка.

Ответ: 2.

\section*{Задание № 3.}

Установить вид формулы алгебры логики:
$$L = ((A \vee B) \wedge C) \to (A \leftrightarrow (B \vee \overline{C} )).$$

\begin{center}Решение:\end{center}

\medskip
\bgroup
\def\arraystretch{1.5}
\setlength\tabcolsep{6}
\begin{tabular}{|c|c|c|c|c|c|c|c|}
\hline
$A$ & $B$ & $C$ & $A \vee B$ & $(A \vee B) \wedge C$ & $B \vee \overline{C}$ & $A\leftrightarrow(B\vee\overline{C})$ & $L$ \\
\hline
0 & 0 & 0 & 0 & 0 & 1 & 0 & 1 \\
\hline
0 & 0 & 1 & 0 & 0 & 0 & 1 & 1 \\
\hline
0 & 1 & 0 & 1 & 0 & 1 & 0 & 1 \\
\hline
0 & 1 & 1 & 1 & 1 & 1 & 0 & 0 \\
\hline
1 & 0 & 0 & 1 & 0 & 1 & 1 & 1 \\
\hline
1 & 0 & 1 & 1 & 1 & 0 & 0 & 0 \\
\hline
1 & 1 & 0 & 1 & 0 & 1 & 1 & 1 \\
\hline
1 & 1 & 1 & 1 & 1 & 1 & 1 & 1 \\
\hline
\end{tabular}
\egroup
\medskip

Из полученной таблицы видно, что формула $L$ является выполнимой, так как она принимает значение $1$, но не является тождественно выполнимой (тавтологией), ибо при определенных значениях высказываний она принимает значение $0$.

\section*{Задание № 4.}

Упростить формулу:

$$\varphi=\overline{A\to B}\wedge\left(B\to\overline{A}\right).$$

Проверить результат, используя таблицу истинности.

\begin{center}Решение:\end{center}

$$\varphi=\overline{A\to B}\wedge\left(B\to\overline{A}\right)=\overline{\overline{A}\vee B}\wedge\left(\overline{B}\vee\overline{A}\right)=\left(\overline{\overline{A}}\wedge\overline{B}\right)\wedge\left(\overline{B}\vee\overline{A}\right)=\left(A\wedge\overline{B}\right)\wedge\left(\overline{B}\vee\overline{A}\right)=$$

$$=A\wedge\overline{B}\wedge\overline{B}\vee A\wedge\overline{B}\wedge\overline{A}=A\wedge\overline{B}\vee A\wedge\overline{B}\wedge\overline{A}=A\wedge\overline{B}\vee0\wedge\overline{B}=A\wedge\overline{B}\vee0=A\wedge\overline{B}.$$


\medskip
\bgroup
\def\arraystretch{1.5}
\setlength\tabcolsep{6}
\begin{tabular}{|c|c|c|c|c|c|c|}
\hline
$A$ & $B$ & $A\to B$ & $\overline{A\to B}$ & $B\to\overline{A}$ & $\varphi=\overline{A\to B}\wedge\left(B\to\overline{A}\right)$ & $A\wedge\overline{B}$ \\
\hline
0 & 0 & 1 & 0 & 1 & 0 & 0 \\
\hline
0 & 1 & 1 & 0 & 1 & 0 & 0 \\
\hline
1 & 0 & 0 & 1 & 1 & 1 & 1 \\
\hline
1 & 1 & 1 & 0 & 0 & 0 & 0 \\
\hline
\end{tabular}
\egroup
\medskip

Значения в последних двух столбцах таблицы истинности совпадают, значит результат упрощения формулы $\varphi$ верен.

\section*{Задание № 5.}

Для орграфа, представленного на рисунке, найти матрицу смежности и матрицу инцидентности. Есть ли у данного графа циклы? Если есть, то приведите пример простого цикла.

\medskip
\begin{tikzpicture}
\begin{scope}[every node/.style={circle,thick,draw}]
    \node (A) at (-3,0) {$1$};
    \node (B) at (0,-2) {$5$};
    \node (C) at (5,-2) {$4$};
    \node (D) at (5,2) {$3$};
    \node (E) at (0,2) {$2$};
\end{scope}

\begin{scope}[>={Stealth[black]},
              every node/.style={fill=white,circle},
              every edge/.style={draw=red,very thick}]
    \path [->] (B) edge node {$e_2$} (A);
    \path [->] (B) edge[bend right=12] node {$e_5$} (E);
    \path [->] (B) edge node {$e_8$} (C);
    \path [->] (E) edge node {$e_3$} (D);
    \path [->] (E) edge[bend right=12] node {$e_4$} (B);
    \path [->] (D) edge[bend left=12] node {$e_7$} (C);
    \path [->] (D) edge[bend right=12] node {$e_6$} (C);
    \path [->] (E) edge node {$e_1$} (A);
\end{scope}
\end{tikzpicture}
\medskip

\begin{center}Решение:\end{center}

Построим матрицы смежности и инцидентности:

$$A=
\begin{pmatrix}
0 & 0 & 0 & 0 & 0 \\
1 & 0 & 1 & 0 & 1 \\
0 & 0 & 0 & 2 & 0 \\
0 & 0 & 0 & 0 & 0 \\
1 & 1 & 0 & 1 & 0 \\
\end{pmatrix};
\quad
B=
\begin{pmatrix}
-1 & -1 & 0 & 0 & 0 & 0 & 0 & 0 \\
1 & 0 & 1 & 1 & -1 & 0 & 0 & 0 \\
0 & 0 & -1 & 0 & 0 & 1 & 1 & 0 \\
0 & 0 & 0 & 0 & 0 & -1 & -1 & -1 \\
0 & 1 & 0 & -1 & 1 & 0 & 0 & 1 \\
\end{pmatrix}.
$$

Данный граф содержит простой цикл $v_2 e_4 v_5 e_5 v_2$.

(Кстати в методичке в определении понятия простой цикл нужно уточнить что вершины хоть и не должны повторяться, но разумеется за исключением начальной и конечной, без этого уточнения определение не корректно.)

\section*{Задание № 6.}

Определить функцию $f(x,y)$, полученную из функций $g(x)=0$ и
$h(x, y, z) = x^2+z$ по схеме примитивной рекурсии.

\begin{center}Решение:\end{center}

Имеем по определению оператора примитивной рекурсии:

1) $f(x,0)=g(x)=0$;

2) $f(x,y+1)=h\left(x,y,f(x,y)\right)=x^2+f(x,y)$.

Отсюда имеем:

$f(x,1)=x^2+f(x,0)=x^2$;

$f(x,2)=x^2+f(x,1)=x^2+x^2=2x^2$;

$f(x,3)=x^2+f(x,2)=x^2+2x^2=3x^2$ и так далее.

Можно предположить, что $f(x,y)=yx^2$, и доказать эту формулу
методом математической индукции по переменной $y$.

\end{document}
