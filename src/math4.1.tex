\documentclass{article}
\usepackage{fullpage}
%%% Работа с русским языком
\usepackage[T2A]{fontenc}
\usepackage[utf8]{inputenc}
\usepackage[english,russian]{babel}   %% загружает пакет многоязыковой вёрстки
\usepackage{indentfirst}
\frenchspacing
\usepackage{titlesec} % package to customize chapters, sections and subsections style
%--------------------------------------
\titleformat{\section}{\large\bfseries\centering}{\thesection}{1em}{}
%Hyphenation rules
%--------------------------------------
\usepackage{hyphenat}
\hyphenation{мате-мати-ка восста-навливать}
%--------------------------------------
\usepackage[intlimits,sumlimits]{amsmath}
\usepackage{amsfonts,amssymb,amsthm,mathtools}
\usepackage[bb=pazo,frak=pxtx,frakscaled=1.3]{mathalfa}
\usepackage{upgreek}
\usepackage{graphicx}
\usepackage{verbatim} %% для многострочных комментариев

%% Перенос знаков в формулах (по Львовскому)
\newcommand*{\hm}[1]{#1\nobreak\discretionary{}
{\hbox{$\mathsurround=0pt #1$}}{}}

\renewcommand{\le}{\ensuremath{\leqslant}}
\renewcommand{\leq}{\ensuremath{\leqslant}}
\renewcommand{\ge}{\ensuremath{\geqslant}}
\renewcommand{\geq}{\ensuremath{\geqslant}}

\newcommand{\tg}{\mathop{\rm tg}\nolimits}

\usepackage{pgfplots}
\pgfplotsset{width=10cm,compat=1.9}
\usepgfplotslibrary{external}
\tikzexternalize
\tikzset{>=latex}

\title{Контрольная работа №4 по курсу «Математика»\\
Тема: «Дифференциальное исчисление функций одной переменной»\\
Вариант №1}
\author{Ригованов Филипп Юрьевич, студент группы КТбз1-1}
\begin{document}
\date{}
\maketitle
\section*{Задание № 1.}
Найти пределы функций в точке $x_0$ , используя свойства пределов, замечательные
пределы и сравнение бесконечно малых.

$$1)\; \lim_{x\to x_0}\frac{2x^2-6x+4}{3x^2-3}=\frac{2}{3}\lim_{x\to x_0}\frac{x^2-3x+2}{x^2-1}=\frac{2}{3}\lim_{x\to x_0}\frac{(x-2)(x-1)}{(x+1)(x-1)}=\frac{2}{3}\lim_{x\to x_0}\frac{x-2}{x+1};$$

а) $x_0=2$; б) $x_0=-1$; в) $x_0=1$; г) $x_0=\infty$.

$$\textit{а})\;\frac{2}{3}\lim_{x\to2}\frac{x-2}{x+1}=\frac{2}{3}\cdot\frac{2-2}{2+1}=\frac{2}{3}\cdot\frac{0}{3}=0;\qquad\textit{б})\;\frac{2}{3}\lim_{x\to{-1}}\frac{x-2}{x+1}=\frac{2}{3}\cdot\frac{-1-2}{-1+1}=\frac{-2}{0}=-\infty;$$

$$\textit{в})\;\frac{2}{3}\lim_{x\to1}\frac{x-2}{x+1}=\frac{2}{3}\frac{1-2}{1+1}=-\frac{1}{3};\qquad\textit{г})\;\frac{2}{3}\lim_{x\to\infty}\frac{x-2}{x+1}=\frac{2}{3}\lim_{x\to\infty}\frac{1-\frac{2}{x}}{1+\frac{1}{x}}=\frac{2}{3}\cdot\frac{1-0}{1+0}=\frac{2}{3}.$$

$$2)\; \lim_{x\to0}\frac{\sqrt{1+x}-\sqrt{1-x}}{3x}=\lim_{x\to0}\frac{(\sqrt{1+x}-\sqrt{1-x})(\sqrt{1+x}+\sqrt{1-x})}{3x(\sqrt{1+x}+\sqrt{1-x})}=\lim_{x\to0}\frac{(1+x-(1-x))}{3x(\sqrt{1+x}+\sqrt{1-x})}=$$

$$=\lim_{x\to0}\frac{2x}{3x(\sqrt{1+x}+\sqrt{1-x})}=\lim_{x\to0}\frac{2}{3(\sqrt{1+x}+\sqrt{1-x})}=\frac{2}{3(\sqrt{1+0}+\sqrt{1-0})}=\frac{1}{3}.$$

$$3)\; \lim_{x\to0}\frac{\cos{x}-\cos^2{x}}{x^2}=\lim_{x\to0}\frac{\cos{x}(1-\cos{x})}{x^2}=\lim_{x\to0}\cos{x}\;\cdot\lim_{x\to0}\frac{1-\cos{x}}{x^2}=1\cdot\lim_{x\to0}\frac{2\sin^2{\frac{x}{2}}}{x^2}=\frac{1}{2}\lim_{x\to0}\left(\frac{\sin{\frac{x}{2}}}{\frac{x}{2}}\right)^2=\frac{1}{2}.$$

$$4)\; \lim_{x\to0}\frac{\left(\sqrt[3]{1+\tg{2x}}-1\right)\ln{(1+\sin^2{3x})}}{(1-\cos{x})\left(2^{\arctg{4x}}-1\right)}=\lim_{x\to0}\frac{\frac{1}{3}\cdot\tg{2x}\cdot\sin^2{3x}}{\frac{x^2}{2}\cdot\arctg{4x}\cdot\ln2}=\lim_{x\to0}\frac{\frac{1}{3}\cdot2x\cdot(3x)^2}{\frac{x^2}{2}\cdot4x\cdot\ln2}=\frac{3}{\ln2}\approx4,328.$$

\section*{Задание № 2.}

Исследовать на непрерывность функции $f(x)$ и построить эскизы их графиков.

\begin{equation*}
\textit{а})\;f(x)=
 \begin{cases}
  x - 1, & \textit{если}\quad x\leq 1,\\
  \sin{(x-1)}, & \textit{если}\quad 1<x\leq 2,\\
  x^2, & \textit{если}\quad x > 2;
 \end{cases}
 \qquad\textit{б})\;f(x)=3^{\frac{4}{(x-2)^2(x^2+5x+4)}}.
\end{equation*}

\begin{center}Решение:\end{center}

а) Функции $x - 1$, $\sin{(x-1)}$, $x^2$ - элементарные, поэтому они непрерывны в области их определения, то есть при любых значениях аргумента $x$ . Следовательно, заданная неэлементарная функция $f(x)$ может иметь точки разрыва только в точках «стыковки» функций $x - 1$, $\sin{(x-1)}$, $x^2$, то есть при $x_1=1$ и $x_2=2$.

Проверим выполнение условий непрерывности функции $f(x)$ в этих точках.


1) $x=1$.

$$\lim_{x\to1-0}f(x)=\lim_{x\to1-0}x-1=1-1=0;\quad\lim_{x\to1+0}f(x)=\lim_{x\to1+0}\sin{(x-1)}=\sin{(1-1)}=\sin0=0;\quad f(1)=1-1=0.$$

Так как $\lim\limits_{x\to1-0}f(x)=\lim\limits_{x\to1+0}f(x)=f(1)$, то функция $f(x)$ непрерывна в точке $x=1$.


1) $x=2$.

$$\lim_{x\to2-0}f(x)=\lim_{x\to2-0}\sin{(x-1)}=\sin{(2-1)}=\sin1\approx0,84;\quad\lim_{x\to1+0}f(x)=\lim_{x\to2+0}x^2=2^2=4.$$

Так как $\lim\limits_{x\to2-0}f(x)\neq\lim\limits_{x\to2+0}f(x)$, то точка $x=2$ является точкой разрыва функции $f(x)$.

Так как односторонние пределы конечны, но не равны, то точка $x=2$ является неустранимой точкой разрыва первого рода.

Построим эскиз графика функции $f(x)$.

\begin{tikzpicture}
\begin{axis}[
  legend pos=outer north east,
  axis x line=center,
  axis y line=center,
  xtick={-5,-4,...,5},
  ytick={-5,-4,...,5},
  xlabel={$x$},
  ylabel={$y$},
  xlabel style={below right},
  ylabel style={above left},
  xmin=-0.5,
  xmax=3.5,
  ymin=-1.5,
  ymax=5.5
]
\addplot [
    domain=-0.5:1,
    samples=100,
    color=red,
]
{x-1};
\addlegendentry{$x-1,\quad x\leq 1$}
\addplot [
    domain=1:2,
    samples=100,
    color=blue,
]
{sin(deg(x-1))};
\addlegendentry{$\sin{(x-1)},\quad 1<x\leq 2$}
\addplot [
    domain=2:3,
    samples=100,
    color=green,
]
{x^2};
\addlegendentry{$x^2,\quad x > 2$}

\draw[blue,->] (axis cs:1.05, 0.05) -- (axis cs:1, 0);
\draw[green,->] (axis cs:2.005, 4.02) -- (axis cs:2, 4);

\end{axis}
\end{tikzpicture}

\section*{Задание № 3.}

Используя определение производной, найти производную функции.

$$y=\sin{(2x+3)}$$

\begin{center}Решение:\end{center}

По определению $y^\prime=\lim\limits_{\Delta x\to0}\frac{\Delta y}{\Delta x}$. Зафиксируем произвольное значение аргумента $x$ и придадим ему приращение $\Delta x$, тогда функция $y=\sin{(2x+3)}$ получит приращение

$$\Delta y=\sin{(2(x+\Delta x)+3)}-\sin{(2x+3)}=2\sin{\frac{2x+2\Delta x+3-2x-3}{2}}\cdot\cos{\frac{2x+2\Delta x+3+2x+3}{2}}=$$

$$=2\sin{\Delta x}\cdot\cos{(2x+\Delta x+3)}.$$

Далее находим

$$y^\prime=\lim\limits_{\Delta x\to0}\frac{\Delta y}{\Delta x}=\lim\limits_{\Delta x\to0}\frac{2\sin{\Delta x}\cdot\cos{(2x+\Delta x+3)}}{\Delta x}=2\lim\limits_{\Delta x\to0}\frac{\sin{\Delta x}}{\Delta x}\cdot\lim\limits_{\Delta x\to0}\cos{(2x+\Delta x+3)}=2\cdot1\cdot\cos{(2x+3)}=$$
$$=2\cos{(2x+3)}.$$

Таким образом $\left(\sin{(2x+3)}\right)^ \prime=2\cos{(2x+3)}$.

\section*{Задание № 4.}

Найти производные $y_x ^\prime=\frac{dy}{dx}$ следующих функций:

$$1)\;y=2\sqrt{4x+3};\qquad2)\;y=(e^{\cos{x}}+1)^2;\qquad3)\;y=3x\cdot\sin{(2x+1)};$$

$$4)\;x=\arcsin{t},\quad y=\sqrt{1-t^2};\qquad\qquad5)\;\ln{x}+e^{-\frac{y}{x}}=\frac{1}{e}.$$

\begin{center}Решение:\end{center}

$$1)\;y^\prime=\left(2\sqrt{4x+3}\right)^\prime=\frac{\left(4x+3\right)^\prime}{\sqrt{4x+3}}=\frac{4}{\sqrt{4x+3}};$$
$$2)\;y^\prime=\left(\left(e^{\cos{x}}+1\right)^2\right)^\prime=2\cdot\left(e^{\cos{x}}+1\right)\cdot\left(e^{\cos{x}}+1\right)^\prime=2\cdot\left(e^{\cos{x}}+1\right)\cdot e^{\cos{x}}\cdot\left(\cos{x}\right)^\prime=-2\sin{x}\cdot\left(e^{\cos{x}}+1\right)\cdot e^{\cos{x}};$$
$$3)\;y^\prime=\left(3x\cdot\sin{(2x+1)}\right)^\prime=\left(3x\right)^\prime\cdot\sin{(2x+1)}+3x\cdot\left(\sin{(2x+1)}\right)^\prime=3\sin{(2x+1)}+3x\cdot\cos{(2x+1)}\cdot(2x+1)^\prime=$$
$$=3\sin{(2x+1)}+6x\cdot\cos{(2x+1)};$$
$$4)\;y_x^\prime=\frac{y_t^\prime}{x_t^\prime}=\frac{\left(\sqrt{1-t^2}\right)^\prime}{(\arcsin{t})^\prime}=\frac{\frac{-t}{\sqrt{1-t^2}}}{\frac{1}{\sqrt{1-t^2}}}=-t;$$
$5)$ Продифференцируем обе части равенства $\ln{x}+e^{-\frac{y}{x}}=\frac{1}{e}$ по $x$, считая $y$ промежуточным аргументом: $\left(\ln{x}+e^{-\frac{y}{x}}\right)^\prime=\left(\frac{1}{e}\right)^\prime$; $\frac{1}{x}+=0$;

\section*{Задание № 5.}

Для заданной функции $y=f(x)$ найти дифференциалы первого и второго порядков $dy$ и $d^2y$.

$$y=\ln{\tg{x}}$$

\begin{center}Решение:\end{center}

\section*{Задание № 6.}

Пользуясь правилом Лопиталя, найти предел.

$$\lim_{x\to0}=\frac{e^{x\sin{x}}-1}{x^2\cos{x}}$$

\begin{center}Решение:\end{center}

\section*{Задание № 7.}

Средствами дифференциального исчисления исследовать функцию $y=\frac{Bx^3}{x^2+Bx-C}$ и построить её график. Значения коэффициентов $B=2$ и $C=3$.

\begin{center}Решение:\end{center}

\end{document}
