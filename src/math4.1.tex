\documentclass{article}
\usepackage{fullpage}
%%% Работа с русским языком
\usepackage[T2A]{fontenc}
\usepackage[utf8]{inputenc}
\usepackage[english,russian]{babel}   %% загружает пакет многоязыковой вёрстки
\usepackage{indentfirst}
\frenchspacing
\usepackage{titlesec} % package to customize chapters, sections and subsections style
%--------------------------------------
\titleformat{\section}{\large\bfseries\centering}{\thesection}{1em}{}
%Hyphenation rules
%--------------------------------------
\usepackage{hyphenat}
\hyphenation{мате-мати-ка восста-навливать}
%--------------------------------------
\usepackage[intlimits,sumlimits]{amsmath}
\usepackage{amsfonts,amssymb,amsthm,mathtools}
\usepackage[bb=pazo,frak=pxtx,frakscaled=1.3]{mathalfa}
\usepackage{upgreek}
\usepackage{graphicx}
\usepackage{verbatim} %% для многострочных комментариев

%% Перенос знаков в формулах (по Львовскому)
\newcommand*{\hm}[1]{#1\nobreak\discretionary{}
{\hbox{$\mathsurround=0pt #1$}}{}}

\renewcommand{\le}{\ensuremath{\leqslant}}
\renewcommand{\leq}{\ensuremath{\leqslant}}
\renewcommand{\ge}{\ensuremath{\geqslant}}
\renewcommand{\geq}{\ensuremath{\geqslant}}

\title{Контрольная работа №4 по курсу «Математика»\\
Тема: «Кратные интегралы»\\
Вариант №1}
\author{Ригованов Филипп Юрьевич, студент группы КТбз1-1}
\begin{document}
\date{}
\maketitle
\section*{Задание № 1.}
Найти пределы функций в точке $x_0$ , используя свойства пределов, замечательные
пределы и сравнение бесконечно малых.

$$1)\; \lim_{x\to x_0}\frac{2x^2-6x+4}{3x^2-3}=\frac{2}{3}\lim_{x\to x_0}\frac{x^2-3x+2}{x^2-1}=\frac{2}{3}\lim_{x\to x_0}\frac{(x-2)(x-1)}{(x+1)(x-1)}=\frac{2}{3}\lim_{x\to x_0}\frac{x-2}{x+1};$$

а) $x_0=2$; б) $x_0=-1$; в) $x_0=1$; г) $x_0=\infty$.

$$\textit{а})\;\frac{2}{3}\lim_{x\to2}\frac{x-2}{x+1}=\frac{2}{3}\cdot\frac{2-2}{2+1}=\frac{2}{3}\cdot\frac{0}{3}=0;\;\textit{б})\;\frac{2}{3}\lim_{x\to{-1}}\frac{x-2}{x+1}=\frac{2}{3}\cdot\frac{-1-2}{-1+1}=\frac{-2}{0}=-\infty;$$

$$\textit{в})\;\frac{2}{3}\lim_{x\to1}\frac{x-2}{x+1}=\frac{2}{3}\frac{1-2}{1+1}=-\frac{1}{3};\;\textit{г})\;\frac{2}{3}\lim_{x\to\infty}\frac{x-2}{x+1}=\frac{2}{3}\lim_{x\to\infty}\frac{1-\frac{2}{x}}{1+\frac{1}{x}}=\frac{2}{3}\cdot\frac{1-0}{1+0}=\frac{2}{3}.$$

$$2)\; \lim_{x\to0}\frac{\sqrt{1+x}-\sqrt{1-x}}{3x}=\lim_{x\to0}\frac{(\sqrt{1+x}-\sqrt{1-x})(\sqrt{1+x}+\sqrt{1-x})}{3x(\sqrt{1+x}+\sqrt{1-x})}=\lim_{x\to0}\frac{(1+x-(1-x))}{3x(\sqrt{1+x}+\sqrt{1-x})}=$$

$$=\lim_{x\to0}\frac{2x}{3x(\sqrt{1+x}+\sqrt{1-x})}=\lim_{x\to0}\frac{2}{3(\sqrt{1+x}+\sqrt{1-x})}=\frac{2}{3(\sqrt{1+0}+\sqrt{1-0})}=\frac{1}{3}.$$

$$3)\; \lim_{x\to0}\frac{\cos{x}-\cos^2{x}}{x^2}=$$

\section*{Задание № 2.}

\begin{center}Решение:\end{center}

\section*{Задание № 3.}

\begin{center}Решение:\end{center}

\section*{Задание № 4.}

\begin{center}Решение:\end{center}

\section*{Задание № 5.}

\begin{center}Решение:\end{center}

\section*{Задание № 6.}

\begin{center}Решение:\end{center}

\end{document}
