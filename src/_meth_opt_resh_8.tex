\documentclass[fleqn]{article}
\usepackage{fullpage}
%%% Работа с русским языком
\usepackage[T2A]{fontenc}
\usepackage[utf8]{inputenc}
\usepackage[english,russian]{babel}   %% загружает пакет многоязыковой вёрстки
\usepackage{indentfirst}
\frenchspacing
\usepackage{titlesec} % package to customize chapters, sections and subsections style
%--------------------------------------
\titleformat{\section}{\large\bfseries\centering}{\thesection}{1em}{}
%Hyphenation rules
%--------------------------------------
\usepackage{hyphenat}
\hyphenation{мате-мати-ка восста-навливать}
%--------------------------------------
\usepackage[intlimits,sumlimits,fleqn]{amsmath}
\usepackage{amsfonts,amssymb,amsthm,mathtools}
\usepackage[bb=pazo,frak=pxtx,frakscaled=1.3]{mathalfa}
\usepackage{upgreek}
\usepackage{graphicx}
\usepackage{verbatim} %% для многострочных комментариев

%% Перенос знаков в формулах (по Львовскому)
\newcommand*{\hm}[1]{#1\nobreak\discretionary{}
{\hbox{$\mathsurround=0pt #1$}}{}}

\renewcommand{\le}{\ensuremath{\leqslant}}
\renewcommand{\leq}{\ensuremath{\leqslant}}
\renewcommand{\ge}{\ensuremath{\geqslant}}
\renewcommand{\geq}{\ensuremath{\geqslant}}

\newcommand{\tg}{\mathop{\rm tg}\nolimits}

\usepackage{tikz}
\usepackage{pgfplots}
\pgfplotsset{width=10cm,compat=1.9}
\usepgfplotslibrary{external}
\tikzexternalize
\tikzset{>=latex}

\usetikzlibrary{arrows.meta}

\usepackage{multirow}

\usepackage{xcolor,colortbl}

\newcommand{\mc}[2]{\multicolumn{#1}{c}{#2}}
\definecolor{Gray}{gray}{0.85}

\newcolumntype{a}{>{\columncolor{Gray}}c}
\newcolumntype{b}{>{\columncolor{white}}c}

\DeclareMathOperator{\sign}{sign}

\title{Контрольная работа по дициплине «Методы оптимальных решений»\\
Вариант №8}

\begin{document}
\date{}
\maketitle

\section*{Задание № 1.}

\begin{center}Решение:\end{center}

\section*{Задание № 2.}

Имеется два вида корма $I$ и $II$, содержащие питательные вещества (витамины) $S_1$, $S_2$ и $S_3$. Содержание числа единиц питательных веществ в $1$ кг каждого вида корма и необходимый минимум питательных веществ приведены в таблице:

\bgroup
\def\arraystretch{1.5}
\setlength\tabcolsep{10}
\begin{center}
\begin{tabular}{|c|c|c|c|c|}
\hline
\multirow{2}{*}{\specialcell{Вид\\товара}} & \multicolumn{2}{c|}{\specialcell{Цена за единицу\\товара, руб.}} & \multicolumn{2}{c|}{\specialcell{Объём продаж,\\тыс. штук}} \\
\cline{2-5}
 & 1 квартал & 2 квартал & 1 квартал & 2 квартал \\
\hline
A & 107 & 110 & 220 & 189 \\
\hline
B & 46 & 44 & 490 & 550 \\
\hline
C & 18 & 20 & 720 & 680 \\
\hline
\end{tabular}
\end{center}
\egroup

Стоимость 1 кг корма I и II соответственно равна 4 и 6 ден. ед.
Необходимо составить дневной рацион, имеющий минимальную стоимость.
Построить экономико-математическую модель задачи, дать необходимые комментарии к ее элементам и получить решение графическим методом. Что произойдет, если решать задачу на максимум и почему?

\begin{center}Решение:\end{center}

\section*{Задание № 3.}

\begin{center}Решение:\end{center}

\section*{Задание № 4.}

\begin{center}Решение:\end{center}

\section*{Задание № 5.}

\begin{center}Решение:\end{center}


\begin{comment}
\end{comment}
\end{document}
