\documentclass{article}
\usepackage{fullpage}
%%% Работа с русским языком
\usepackage[T2A]{fontenc}
\usepackage[utf8]{inputenc}
\usepackage[english,russian]{babel}   %% загружает пакет многоязыковой вёрстки
%\usepackage{fontspec}      %% подготавливает загрузку шрифтов Open Type, True Type и др.
%\defaultfontfeatures{Ligatures={TeX},Renderer=Basic}  %% свойства шрифтов по умолчанию
%\setmainfont[Ligatures={TeX,Historic}]{Times New Roman} %% задаёт основной шрифт документа
%\setsansfont{Comic Sans MS}                    %% задаёт шрифт без засечек
%\setmonofont{Courier New}
\usepackage{indentfirst}
\frenchspacing
\usepackage{titlesec} % package to customize chapters, sections and subsections style
%--------------------------------------
\titleformat{\section}{\large\bfseries\centering}{\thesection}{1em}{}
%Hyphenation rules
%--------------------------------------
\usepackage{hyphenat}
\hyphenation{мате-мати-ка восста-навливать}
%--------------------------------------
\usepackage{xcolor}
\usepackage{listingsutf8}
\lstset{
    language=[WinXP]{command.com},
    basicstyle=\ttfamily\small,
    numberstyle=\footnotesize,
    numbers=left,
    backgroundcolor=\color{gray!10},
    frame=single,
    tabsize=2,
    rulecolor=\color{black!30},
    title=\lstname,
    escapeinside={\%*}{*)},
    breaklines=true,
    breakatwhitespace=true,
    framextopmargin=2pt,
    framexbottommargin=2pt,
    extendedchars=false,
    inputencoding=utf8
}

\title{Лабораторная работа № 1 по курсу операционные системы \\
Тема: «Работа с командной строкой. Создание сценариев»}
\author{Ригованов Филипп Юрьевич, студент группы КТбз1-1}
\date{Декабрь 2015}

\begin{document}

\maketitle
\section*{Вариант № 1.}
Разработать пакетный bat-­файл файл для обновления архива. Выбор архиватора осуществляется из меню. Имя архива передается в командной строке.
В пакетном файле предусмотреть сообщение имени, назначения, применения и автора пакетного файла (при пустой командной строке и по ключу~/?), контроль верности командной строки, наличие требуемых файлов и сохранность имени пакетного файла. Текущий каталог не изменять, если это специально не оговорено. Там, где необходимо, имена файлов указывать с полным путем и диском. С клавиатуры при работе пакетного файла вводить только числа, строковые данные выбирать либо из меню, либо передавать в командной строке.

Листинг исходного кода программы:
\begin{lstlisting}
@echo off
rem %*Проверка наличия параметра командной строки*)
if "%1"=="" goto info
if "%1"=="/?" goto info
rem %*Проверка существования каталога*)
if exist "C:\text" (
rem %*Вывод меню на экран*)
cls
echo %*Выберите архиватор:*)
echo 1 - 7zip
echo 2 - WinRaR
@choice /C:12
rem %*Определение сделанного выбора*)
if errorlevel 2 goto winrar
if errorlevel 1 goto 7zip
echo %*Выбор не был сделан.*)
goto end
) else (
echo %*Каталог*) "C:\text" %*отсутсвует*).
goto end
)

:7zip
"C:\Program Files\7-Zip\7z.exe" a -r "%1" C:\text
goto end

:winrar
"C:\Program Files\WinRaR\Rar.exe" a -r "%1" C:\text
goto end

:info
echo %*Скрипт arch.bat предназначен для обновления архива содержимого*)
echo %*каталога*) "C:\text" %*с помощью архиватора 7-Zip или WinRaR на выбор.*)
echo %*Автор: Ригованов Филипп Юрьевич, группа КТбз1-1.*)
echo %*Использование:*)
echo %*Обрабатывается только перый аргумент, если он отсутсвует или равен*) "/?",
echo %*то выводится данная справка, любое другое значение аргумента используется*)
echo %*в качестве имени создаваемого или обновляемого архива.*)
echo %*Если отсутсвует каталог*) "C:\text", %* то выводится сообщение об его отсутствии.*)
pause

:end
\end{lstlisting}
Если нам нужно обновить архив и присвоить ему имя \textbf{test}, то нужно выполнить команду:
\begin{lstlisting}[language={},numbers={}]
C:\>arch.bat test
\end{lstlisting}
\vfill
Результат работы программы при выборе архиватора 7zip:
\begin{lstlisting}[language={},numbers={}]
%*Выберите архиватор:*)
1 - 7zip
2 - WinRaR
[1,2]?1

7-Zip [32] 15.12 : Copyright (c) 1999-2015 Igor Pavlov : 2015-11-19

Open archive: test.7z
--
Path = test.7z
Type = 7z
Physical Size = 354
Headers Size = 190
Method = LZMA2:13
Solid = -
Blocks = 1

Scanning the drive:
1 folder, 3 files, 6148 bytes (7 KiB)

Updating archive: test.7z

Items to compress: 4


Files read from disk: 1
Archive size: 354 bytes (1 KiB)
Everything is Ok

C:\>
\end{lstlisting}
\newpage
Результат работы программы при выборе архиватора WinRaR:
\begin{lstlisting}[language={},numbers={}]
%*Выберите архиватор:*)
1 - 7zip
2 - WinRaR
[1,2]?2

RAR 5.20   Copyright (c) 1993-2014 Alexander Roshal   2 Dec 2014
Trial version             Type RAR -? for help

Evaluation copy. Please register.

Updating archive test.rar

Updating  C:\text\.DS_Store                                           OK
Updating  C:\text\1.txt                                               OK
Updating  C:\text\2.txt                                               OK
Updating  C:\text                                                    100%
Done

C:\>
\end{lstlisting}
Вывод справки:
\begin{lstlisting}[language={},numbers={}]
C:\>arch.bat /?
%*Скрипт arch.bat предназначен для обновления архива содержимого*)
%*каталога*) "C:\text" %*с помощью архиватора 7-Zip или WinRaR на выбор.*)
%*Автор: Ригованов Филипп Юрьевич, группа КТбз1-1.*)
%*Использование:*)
%*Обрабатывается только перый аргумент, если он отсутсвует или равен*) "/?",
%*то выводится данная справка, любое другое значение аргумента используется*)
%*в качестве имени создаваемого или обновляемого архива.*)
%*Если отсутсвует каталог*) "C:\text", %* то выводится сообщение об его отсутствии.*)
%*Нажмите любую клавишу для продолжения...*)
\end{lstlisting}
Результат работы программы в случае отсутствия каталога, который должен архивироваться:
\begin{lstlisting}[language={},numbers={}]
C:\>arch.bat test
%*Каталог*) "C:\text" %*отсутсвует*).

C:\>
\end{lstlisting}
\end{document}
