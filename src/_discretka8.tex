\documentclass[fleqn]{article}
\usepackage{fullpage}
%%% Работа с русским языком
\usepackage[T2A]{fontenc}
\usepackage[utf8]{inputenc}
\usepackage[english,russian]{babel}   %% загружает пакет многоязыковой вёрстки
\usepackage{indentfirst}
\frenchspacing
\usepackage{titlesec} % package to customize chapters, sections and subsections style
%--------------------------------------
\titleformat{\section}{\large\bfseries\centering}{\thesection}{1em}{}
%Hyphenation rules
%--------------------------------------
\usepackage{hyphenat}
\hyphenation{мате-мати-ка восста-навливать}
%--------------------------------------
\usepackage[intlimits,sumlimits,fleqn]{amsmath}
\usepackage{amsfonts,amssymb,amsthm,mathtools}
\usepackage[bb=pazo,frak=pxtx,frakscaled=1.3]{mathalfa}
\usepackage{upgreek}
\usepackage{graphicx}
\usepackage{verbatim} %% для многострочных комментариев

%% Перенос знаков в формулах (по Львовскому)
\newcommand*{\hm}[1]{#1\nobreak\discretionary{}
{\hbox{$\mathsurround=0pt #1$}}{}}

\renewcommand{\le}{\ensuremath{\leqslant}}
\renewcommand{\leq}{\ensuremath{\leqslant}}
\renewcommand{\ge}{\ensuremath{\geqslant}}
\renewcommand{\geq}{\ensuremath{\geqslant}}

\newcommand{\tg}{\mathop{\rm tg}\nolimits}

\usepackage{tikz}
\usepackage{pgfplots}
\pgfplotsset{width=10cm,compat=1.9}
\usepgfplotslibrary{external}
\tikzexternalize
\tikzset{>=latex}

\usetikzlibrary{arrows.meta}

\usepackage{xcolor,colortbl}

\newcommand{\mc}[2]{\multicolumn{#1}{c}{#2}}
\definecolor{Gray}{gray}{0.85}

\newcolumntype{a}{>{\columncolor{Gray}}c}
\newcolumntype{b}{>{\columncolor{white}}c}

\DeclareMathOperator{\sign}{sign}

\title{Контрольная работа по курсу «Элементы дискретной математики»\\
Вариант №8}

\begin{document}
\date{}
\maketitle
\section*{Задание № 1.}

Даны множества чисел: $A=\{2,3,5,6\}$, $B=\{5,6,7,8\}$, $C=\{3,4,6,8\}$ и универсальное множество $U=\{2,3,4,5,6,7,8,9\}$.

Найти множества чисел $D=(B\cap C)\cup \overline{A\cup C},\;\; E=((B\cup \overline{C})\setminus A)\cup(C\cap B)$. Являются ли множества $E$ и $D$ равными? эквивалентными? включающими одно в другое ( $D \subset E$ или $E \subset D$ )? пересекающимися, но не включающими одно в другое? непересекающимися $(D\cap E=\varnothing)$?

\begin{center}Решение:\end{center}

Для нахождения множества $D$ сначала найдем: пересечение $B\cap C=\{6,8\}$, объединение $A\cup C=\{2,3,4,5,6,8\}$, и его дополнение (до множества $U$) $\overline{A\cup C}=\{7,9\}$. Теперь $D=\{6,8\}\cup\{7,9\}=\{6,7,8,9\}$.

Для нахождения множества $E$ сначала найдем: дополнение $C$ (до множества $U$) $\overline{C}=\{2,5,7,9\}$, затем объединение $B\cup \overline{C}=\{2,5,6,7,8,9\}$, разность множеств $(B\cup \overline{C})\setminus A=\{7,8,9\}$, затем пересечение $C\cap B=\{6,8\}$. Теперь $E=\{7,8,9\}\cup\{6,8\}=\{6,7,8,9\}$.

Множества $D$ и $E$ - равные так состоят из одних и тех же элементов.

\section*{Задание № 2.}

Согласно опросу $250$ телезрителей $95$ из них нравится смотреть новости, $125$ предпочитают смотреть спорт, $125$ – комедии, $25$ – новости и комедии, $45$ – спорт и комедии, $35$ – новости и спорт, $5$ любят смотреть три вида программ. Сколько телезрителей смотрят спорт и комедии, но не смотрят новости?

Решить задачу, используя теорию множеств.

\begin{center}Решение:\end{center}

Пусть $|A|$, $|B|$ и $|C|$ --- число телезрителей, предпочитающих смотреть новости, спорт и комедии соответственно. По условию $|A|=95$, $|B|=125$, $|C|=125$, $|A\cap B|=35$, $|A\cap C|=25$, $|B\cap C|=45$, $|A\cap B\cap C|=5$. Телезрители, которым нравятся и спорт и комедии (их число нам известно из условия), бывают двух типов: 1) которым нравятся новости и 2) которые не смотрят новости. Получается что телезрители первого типа любят смотреть три вида программ, а их число нам известно, значит число телезрителей второго типа (которое и требуется найти в задаче) найдём очень просто:

$$|(B\cap C)\setminus A|=|B\cap C|-|A\cap B\cap C|=45-5=40.$$


\section*{Задание № 3.}

Установить вид формулы алгебры логики:
$$L = ((A \leftrightarrow B) \wedge C) \vee (A \to (B \vee \overline{C} )).$$

\begin{center}Решение:\end{center}

\medskip
\begin{tabular}{|c|c|c|c|c|c|c|c|}
\hline
$A$ & $B$ & $C$ & $A \leftrightarrow B$ & $(A \leftrightarrow B) \wedge C$ & $B \vee \overline{C}$ & $A\to(B\vee\overline{C})$ & $L$ \\
\hline
0 & 0 & 0 & 1 & 0 & 1 & 1 & 1 \\
\hline
0 & 0 & 1 & 1 & 1 & 0 & 1 & 1 \\
\hline
0 & 1 & 0 & 0 & 0 & 1 & 1 & 1 \\
\hline
0 & 1 & 1 & 0 & 0 & 1 & 1 & 1 \\
\hline
1 & 0 & 0 & 0 & 0 & 1 & 1 & 1 \\
\hline
1 & 0 & 1 & 0 & 0 & 0 & 0 & 0 \\
\hline
1 & 1 & 0 & 1 & 0 & 1 & 1 & 1 \\
\hline
1 & 1 & 1 & 1 & 1 & 1 & 1 & 1 \\
\hline
\end{tabular}
\medskip

Из полученной таблицы видно, что формула $L$ является выполнимой, так как она принимает значение $1$, но не является тождественно выполнимой (тавтологией), ибо при $A=1, B=0, C=1$ она принимает значение $0$.

\section*{Задание № 4.}

С помощью таблицы истинности найти СДНФ и СКНФ булевой
функции $f(x_1,x_2)=x_1 \vee x_2 \to(x_1 \leftrightarrow x_2)$.

\begin{center}Решение:\end{center}

\medskip
\begin{tabular}{|c|c|c|c|c|}
\hline
$x_1$ & $x_2$ & $x_1 \vee x_2$ & $x_1 \leftrightarrow x_2$ & $f(x_1,x_2)$ \\
\hline
0 & 0 & 0 & 1 & 1 \\
\hline
0 & 1 & 1 & 0 & 0 \\
\hline
1 & 0 & 1 & 0 & 0 \\
\hline
1 & 1 & 1 & 1 & 1 \\
\hline
\end{tabular}
\medskip

1) Функция $f(x_1, x_2)$ равна $1$ на наборах $(x_1, x_2)$: $(0;0)$ и $(1;1)$, т.е. соответствующие конъюнкции (над равными $0$ переменными ставим знак отрицания) $\overline{x_1} \wedge \overline{x_2}$ и $x_1 \wedge x_2$. Соединяя их знаками дизъюнкции, получим СДНФ функции:

$$f(x_1, x_2)=(\overline{x_1}\wedge\overline{x_2})\vee(x_1\wedge x_2).$$

2) Функция $f(x_1, x_2)$ равна $0$ на наборах $(x_1, x_2)$: $(0;1)$ и $(1;0)$, т.е. соответствующие дизъюнкции (над равными $1$ переменными ставим знак отрицания) $x_1 \vee \overline{x_2}$ и $\overline{x_1} \vee x_2$. Соединяя их знаками конъюнкции, получим СКНФ функции:

$$f(x_1, x_2)=(x_1\vee\overline{x_2})\wedge(\overline{x_1}\vee x_2).$$

\section*{Задание № 5.}

Даны матрицы:

$$A=
\begin{pmatrix}
0 & 1 & 0 & 1 & 1 \\
1 & 1 & 1 & 0 & 1 \\
0 & 1 & 0 & 1 & 0 \\
1 & 0 & 1 & 0 & 1 \\
1 & 1 & 0 & 1 & 0 \\
\end{pmatrix}
\quad
\textit{и}
\quad
B=
\begin{pmatrix}
1 & 1 & 1 & 0 & 0 & 0 & 0 & 0 \\
1 & 0 & 0 & 1 & 1 & 0 & 0 & 0 \\
0 & 0 & 0 & 1 & 0 & 1 & 0 & 0 \\
0 & 1 & 0 & 0 & 0 & 0 & 1 & 1 \\
0 & 0 & 1 & 0 & 1 & 0 & 1 & 1 \\
\end{pmatrix}.
$$

Построить неориентированные графы, для которых матрица $A$ является
матрицей смежности, а матрица $B$ – матрицей инцидентности.

\begin{center}Решение:\end{center}

Так как матрица $A$ имеет размерость равную пяти, то граф должен иметь пять вершин. Единица на диагонали говорит о том что вторая вершина будет смежной сама с собой, значит в этой вершине петля. Изобразим граф для которого матрица $A$ является матрицей смежности:

\medskip
\begin{tikzpicture}
\begin{scope}[every node/.style={circle,thick,draw}]
    \node (A) at (-3,3) {$v_1$};
    \node (B) at (3,3) {$v_2$};
    \node (C) at (4,-2) {$v_3$};
    \node (D) at (0,-4) {$v_4$};
    \node (E) at (-4,-2) {$v_5$};
\end{scope}

\begin{scope}[-,
              every node/.style={fill=white,circle},
              every edge/.style={draw=red,very thick},
              every loop/.style={}]
    \path (A) edge node {$x_1$} (B);
    \path (A) edge[bend left=12] node[pos=0.75] {$x_2$} (D);
    \path (A) edge node {$x_3$} (E);
    \path (B) edge node {$x_4$} (A);
    \path (B) edge[loop above] node {$x_5$} (B);
    \path (B) edge[bend left=12] node {$x_6$} (C);
    \path (B) edge[bend left=12] node {$x_7$} (E);
    \path (C) edge[bend left=12] node {$x_8$} (B);
    \path (C) edge[bend left=12] node {$x_9$} (D);
    \path (D) edge[bend left=12] node[pos=0.75] {$x_{10}$} (A);
    \path (D) edge[bend left=12] node {$x_{11}$} (C);
    \path (D) edge node {$x_{12}$} (E);
    \path (E) edge node {$x_{13}$} (A);
    \path (E) edge[bend left=12] node[pos=0.75] {$x_{14}$} (B);
    \path (E) edge node {$x_{15}$} (D);
\end{scope}
\end{tikzpicture}
\medskip

Так как матрица $B$ имеет пять строк и восемь столбцов, то граф должен иметь пять вершин и весемь рёбер. Единицы в столбцах соответствуют концам одного ребра. Так как в шестом столбце всего одна единица, то шестое ребро должно быть петлёй. Изобразим граф для которого матрица $B$ является матрицей инцидентности:

\medskip
\begin{tikzpicture}
\begin{scope}[every node/.style={circle,thick,draw}]
    \node (A) at (-3,3) {$v_1$};
    \node (B) at (3,3) {$v_2$};
    \node (C) at (4,-2) {$v_3$};
    \node (D) at (0,-4) {$v_4$};
    \node (E) at (-4,-2) {$v_5$};
\end{scope}

\begin{scope}[-,
              every node/.style={fill=white,circle},
              every edge/.style={draw=red,very thick},
              every loop/.style={}]
    \path (A) edge node {$x_1$} (B);
    \path (A) edge node[pos=0.75] {$x_2$} (D);
    \path (A) edge node {$x_3$} (E);
    \path (B) edge node {$x_4$} (C);
    \path (B) edge node {$x_5$} (E);
    \path (C) edge[loop below] node {$x_6$} (C);
    \path (D) edge[bend left=12] node {$x_7$} (E);
    \path (E) edge[bend left=12] node {$x_8$} (D);
\end{scope}
\end{tikzpicture}
\medskip

\section*{Задание № 6.}

Определить функцию $f(x,y)$, полученную из функций $g(x) =1$ и
$h(x, y, z) = \frac{x}{z}$ по схеме примитивной рекурсии.

\begin{center}Решение:\end{center}

Имеем по определению оператора примитивной рекурсии:

1) $f(x,0)=g(x)=1$;

2) $f(x,y+1)=h\left(x,y,f(x,y)\right)=\frac{x}{f(x,y)}$.

Отсюда имеем:

$$f(x,1)=\frac{x}{f(x,0)}=\frac{x}{1}=x;\quad f(x,2)=\frac{x}{f(x,1)}=\frac{x}{x}=1;\quad f(x,3)=\frac{x}{f(x,2)}=\frac{x}{1}=x;\quad\cdots$$

Можно предположить, что $f(x,y)=
\begin{cases}
1 & , \textit{когда} \;y - \textit{чётное}, \\
x & , \textit{когда} \;y - \textit{нечётное}.
\end{cases}$ и доказать эту формулу
методом математической индукции по переменной $y$.

\end{document}
