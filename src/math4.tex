\documentclass{article}
\usepackage{fullpage}
%%% Работа с русским языком
\usepackage[T2A]{fontenc}
\usepackage[utf8]{inputenc}
\usepackage[english,russian]{babel}   %% загружает пакет многоязыковой вёрстки
\usepackage{indentfirst}
\frenchspacing
\usepackage{titlesec} % package to customize chapters, sections and subsections style
%--------------------------------------
\titleformat{\section}{\large\bfseries\centering}{\thesection}{1em}{}
%Hyphenation rules
%--------------------------------------
\usepackage{hyphenat}
\hyphenation{мате-мати-ка восста-навливать}
%--------------------------------------
\usepackage[intlimits,sumlimits]{amsmath}
\usepackage{amsfonts,amssymb,amsthm,mathtools}
\usepackage[bb=pazo,frak=pxtx,frakscaled=1.3]{mathalfa}
\usepackage{upgreek}
\usepackage{graphicx}

%% Перенос знаков в формулах (по Львовскому)
\newcommand*{\hm}[1]{#1\nobreak\discretionary{}
{\hbox{$\mathsurround=0pt #1$}}{}}

\renewcommand{\le}{\ensuremath{\leqslant}}
\renewcommand{\leq}{\ensuremath{\leqslant}}
\renewcommand{\ge}{\ensuremath{\geqslant}}
\renewcommand{\geq}{\ensuremath{\geqslant}}

\title{Контрольная работа №4 по курсу «Математика»\\
Тема: «Кратные интегралы»\\
Вариант №1}
\author{Ригованов Филипп Юрьевич, студент группы КТбз1-1}
\begin{document}
\date{}
\maketitle
\section*{Задание № 1.}
Изменить порядок интегрирования в повторном интеграле. Изобразить область интегрирования.

$$\int\limits_{0}^{1} dy \int\limits_{\sqrt{y}-1}^{1-y} f(x,y)\;dx$$
\begin{center}Решение:\end{center}

$$\int\limits_{0}^{1} dy \int\limits_{\sqrt{y}-1}^{1-y} f(x,y)\;dx=\int\limits_{-1}^{0} dx \int\limits_{0}^{(x+1)^2} f(x,y)\;dy+\int\limits_{0}^{1} dx \int\limits_{0}^{1-x} f(x,y)\;dy.$$

\section*{Задание № 2.}
Вычислить двойной интеграл $\iint\limits_{D} f(x,y)dx dy$ по области $D$, ограниченной заданными кривыми. $f(x,y)=3x^2y+2x,\quad D: y^3=x,\; 4y=x,\; x\geq0$.
\begin{center}Решение:\end{center}

$$\iint\limits_{D} \left(3x^2y+2x\right) dx dy=\frac{1}{2}x^2 y (x y+2)=
\int\limits_{0}^{2} dy \int\limits_{y^3}^{4y} \left(3x^2y+2x\right)\;dx=
\int\limits_{0}^{2} \left(\left(x^3y+x^2\right)\bigg|\limits_{y^3}^{4y}\right) dy =$$
$$= \int\limits_{0}^{2} \left( 64y^4+16y^2-y^{10}-y^6\right) dy = \left(\frac{64y^5}{5}+\frac{16y^3}{3}-\frac{y^{11}}{11}-\frac{y^7}{7}\right)\bigg|\limits_{0}^{2} = \frac{2^{11}}{5}+\frac{2^7}{3}-\frac{2^{11}}{11}-\frac{2^7}{7}=$$

$$=\frac{286208}{1155}\approx247.799.$$

\section*{Задание № 3.}
С помощью двойного интеграла найти площадь фигуры, определенной в полярных координатах указанными неравенствами. $\rho\leq 3\sin{2\varphi}$.
\begin{center}Решение:\end{center}


\section*{Задание № 4.}
 Вычислить тройной интеграл $\iiint\limits_{V} f(x,y,z)dx dy dz$ от заданной функции $f(x,y,z)$ по области $V$, ограниченной указанными поверхностями.
 $f(x,y,z)=2x+y-z,\quad V: 2x+y-z-2=0,\;x=0,\;\\y=0,\;z=0$.
 \begin{center}Решение:\end{center}



\section*{Задание № 5.}
Вычислить тройной интеграл $\iiint\limits_{V} f(x,y,z)dx dy dz$, перейдя к сферической системе координат, где $V$-область, ограниченная указанными поверхностями.
$f(x,y,z)=\frac{1}{z},\quad V: x^2+y^2+z^2=1,\;\\x^2+y^2+z^2=4,\;z=1,\;x\geq0,\;y\geq0,\;z\geq0$.
\begin{center}Решение:\end{center}



\end{document}
