\documentclass[fleqn]{article}
\usepackage{fullpage}
%%% Работа с русским языком
\usepackage[T2A]{fontenc}
\usepackage[utf8]{inputenc}
\usepackage[english,russian]{babel}   %% загружает пакет многоязыковой вёрстки
\usepackage{indentfirst}
\frenchspacing
\usepackage{titlesec} % package to customize chapters, sections and subsections style
%--------------------------------------
\titleformat{\section}{\large\bfseries\centering}{\thesection}{1em}{}
%Hyphenation rules
%--------------------------------------
\usepackage{hyphenat}
\hyphenation{мате-мати-ка восста-навливать}
%--------------------------------------
\usepackage[intlimits,sumlimits,fleqn]{amsmath}
\usepackage{amsfonts,amssymb,amsthm,mathtools}
\usepackage[bb=pazo,frak=pxtx,frakscaled=1.3]{mathalfa}
\usepackage{upgreek}
\usepackage{graphicx}
\usepackage{verbatim} %% для многострочных комментариев

%% Перенос знаков в формулах (по Львовскому)
\newcommand*{\hm}[1]{#1\nobreak\discretionary{}
{\hbox{$\mathsurround=0pt #1$}}{}}

\renewcommand{\le}{\ensuremath{\leqslant}}
\renewcommand{\leq}{\ensuremath{\leqslant}}
\renewcommand{\ge}{\ensuremath{\geqslant}}
\renewcommand{\geq}{\ensuremath{\geqslant}}

\newcommand{\tg}{\mathop{\rm tg}\nolimits}

\usepackage{tikz}
\usepackage{pgfplots}
\pgfplotsset{width=10cm,compat=1.9}
\usepgfplotslibrary{external}
\tikzexternalize
\tikzset{>=latex}

\usetikzlibrary{arrows.meta}

\usepackage{xcolor,colortbl}

\newcommand{\mc}[2]{\multicolumn{#1}{c}{#2}}
\definecolor{Gray}{gray}{0.85}

\newcolumntype{a}{>{\columncolor{Gray}}c}
\newcolumntype{b}{>{\columncolor{white}}c}

\DeclareMathOperator{\sign}{sign}

\title{Контрольная работа по курсу «Теория графов»\\
Вариант №8}

\begin{document}
\date{}
\maketitle
\section*{Задание № 1.}

По матрице смежности восстановите ориентированный граф $D$,
взяв в качестве вершин $v_1$, $v_2$, $v_3$, $v_4$, $v_5$ пять произвольных точек плоскости. Найдите:

1) матрицу инцидентности $B$, предварительно перенумеровав ребра;

2) матрицу достижимости $T$;

3) матрицу сильной связности;

4) компоненты сильной связности.

$$A = \begin{pmatrix}
0 & 0 & 0 & 0 & 1 \\
1 & 0 & 0 & 0 & 0 \\
0 & 1 & 0 & 0 & 0 \\
0 & 1 & 1 & 0 & 1 \\
0 & 0 & 1 & 0 & 0 \\
\end{pmatrix}
$$

\begin{center}Решение:\end{center}

Сначала восстановим ориентированный граф и пронумеруем его ребра.

\medskip

\begin{tikzpicture}
\begin{scope}[every node/.style={circle,thick,draw}]
    \node (A) at (-2,2) {$v_1$};
    \node (B) at (2,2) {$v_2$};
    \node (C) at (2,-2) {$v_3$};
    \node (D) at (0,0) {$v_4$};
    \node (E) at (-2,-2) {$v_5$};
\end{scope}

\begin{scope}[>={Stealth[black]},
              every node/.style={fill=white,circle},
              every edge/.style={draw=red,very thick}]
    \path [->] (A) edge node {$x_1$} (E);
    \path [->] (B) edge node {$x_2$} (A);
    \path [->] (C) edge node {$x_3$} (B);
    \path [->] (D) edge node {$x_4$} (B);
    \path [->] (D) edge node {$x_5$} (C);
    \path [->] (D) edge node {$x_6$} (E);
    \path [->] (E) edge node {$x_7$} (C);
\end{scope}
\end{tikzpicture}

1) Тогда матрицей инцидентности ориентированного графа $D$ будет

$$B(D)=
\begin{pmatrix}
-1 & 1 & 0 & 0 & 0 & 0 & 0 \\
0 & -1 & 1 & 1 & 0 & 0 & 0 \\
0 & 0 & -1 & 0 & 1 & 0 & 1 \\
0 & 0 & 0 & -1 & -1 & -1 & 0 \\
1 & 0 & 0 & 0 & 0 & 1 & -1 \\
\end{pmatrix}
$$
\newpage
2) Найдем матрицу достижимости ориентированного графа $D$:

$$
\sign{A^2} = \begin{pmatrix}
0 & 0 & 1 & 0 & 0 \\
0 & 0 & 0 & 0 & 1 \\
1 & 0 & 0 & 0 & 0 \\
1 & 1 & 1 & 0 & 0 \\
0 & 1 & 0 & 0 & 0 \\
\end{pmatrix}
;\quad
\sign{A^3} = \begin{pmatrix}
0 & 1 & 0 & 0 & 0 \\
0 & 0 & 1 & 0 & 0 \\
0 & 0 & 0 & 0 & 1 \\
1 & 1 & 0 & 0 & 1 \\
1 & 0 & 0 & 0 & 0 \\
\end{pmatrix}
;\quad
\sign{A^4} = \begin{pmatrix}
1 & 0 & 0 & 0 & 0 \\
0 & 1 & 0 & 0 & 0 \\
0 & 0 & 1 & 0 & 0 \\
1 & 0 & 1 & 0 & 1 \\
0 & 0 & 0 & 0 & 1 \\
\end{pmatrix}.
$$

$T(D)= \sign{[E + A + A^2 + A^3 + A^4]}=$
$$=\sign\left[\begin{pmatrix}
1 & 0 & 0 & 0 & 0\\
0 & 1 & 0 & 0 & 0\\
0 & 0 & 1 & 0 & 0\\
0 & 0 & 0 & 1 & 0\\
0 & 0 & 0 & 0 & 1\\
\end{pmatrix}
+
\begin{pmatrix}
0 & 0 & 0 & 0 & 1 \\
1 & 0 & 0 & 0 & 0 \\
0 & 1 & 0 & 0 & 0 \\
0 & 1 & 1 & 0 & 1 \\
0 & 0 & 1 & 0 & 0 \\
\end{pmatrix}
+
\begin{pmatrix}
0 & 0 & 1 & 0 & 0 \\
0 & 0 & 0 & 0 & 1 \\
1 & 0 & 0 & 0 & 0 \\
1 & 1 & 1 & 0 & 0 \\
0 & 1 & 0 & 0 & 0 \\
\end{pmatrix}
+
\begin{pmatrix}
0 & 1 & 0 & 0 & 0 \\
0 & 0 & 1 & 0 & 0 \\
0 & 0 & 0 & 0 & 1 \\
1 & 1 & 0 & 0 & 1 \\
1 & 0 & 0 & 0 & 0 \\
\end{pmatrix}
+
\begin{pmatrix}
1 & 0 & 0 & 0 & 0 \\
0 & 1 & 0 & 0 & 0 \\
0 & 0 & 1 & 0 & 0 \\
1 & 0 & 1 & 0 & 1 \\
0 & 0 & 0 & 0 & 1 \\
\end{pmatrix}\right]
=$$
$$=\begin{pmatrix}
1 & 1 & 1 & 0 & 1\\
1 & 1 & 1 & 0 & 1\\
1 & 1 & 1 & 0 & 1\\
1 & 1 & 1 & 1 & 1\\
1 & 1 & 1 & 0 & 1\\
\end{pmatrix}
$$

3) Найдем матрицу сильной связности ориентированного графа $D$:

$$S(D)= T(D) \:\&\: T^T(D)=
\begin{pmatrix}
1 & 1 & 1 & 0 & 1\\
1 & 1 & 1 & 0 & 1\\
1 & 1 & 1 & 0 & 1\\
1 & 1 & 1 & 1 & 1\\
1 & 1 & 1 & 0 & 1\\
\end{pmatrix}
\:\&\:
\begin{pmatrix}
1 & 1 & 1 & 1 & 1\\
1 & 1 & 1 & 1 & 1\\
1 & 1 & 1 & 1 & 1\\
0 & 0 & 0 & 1 & 0\\
1 & 1 & 1 & 1 & 1\\
\end{pmatrix}
=\begin{pmatrix}
1 & 1 & 1 & 0 & 1\\
1 & 1 & 1 & 0 & 1\\
1 & 1 & 1 & 0 & 1\\
0 & 0 & 0 & 1 & 0\\
1 & 1 & 1 & 0 & 1\\
\end{pmatrix}.
$$

4) Выделим компоненты связности ориентированного графа $D$.

Присваиваем $p=1$, $S_1 = S(D)$ и составляем множество вершин
первой компоненты сильной связности $D_1$: это те вершины, которым соответствуют единицы в первой строке матрицы $S(D)$. Таким образом, первая компонента сильной связности состоит из четырёх вершин $V_1=\{v_1, v_2, v_3, v_4\}$. Составляем матрицу смежности для компоненты сильной связности $D_1$ исходного графа $D$. Для этого возьмем подматрицу матрицы $A(D)$, состоящую из элементов матрицы $A$, находящихся на пересечении строк и столбцов, соответствующих вершинам из $V_1$:

$$A(D_1)=
\begin{pmatrix}
0 & 0 & 0 & 1\\
1 & 0 & 0 & 0\\
0 & 1 & 0 & 0\\
0 & 0 & 1 & 0\\
\end{pmatrix}
$$

Вычеркиваем из матрицы $S_1(D)$ строки и столбцы,
соответствующие вершинам $v_1$, $v_2$, $v_3$ и $v_5$, чтобы получить матрицу $S_2(D)$, которая состоит из одного элемента: $S_2(D)=(1)$
и присваиваем $p=2$. Значит вторая компонента сильной связности исходного графа, состоит из одной вершины $V_2 = \{v_4\}$.

Таким образом, выделены $2$ компоненты сильной связности ориентированного графа $D$:

\medskip

\begin{tikzpicture}
\begin{scope}
    \node (D1) at (-2,3) {$D_1:$};
    \node (D2) at (5,3) {$D_2:$};
\end{scope}
\begin{scope}[every node/.style={circle,thick,draw}]
    \node (A) at (-2,2) {$v_1$};
    \node (B) at (2,2) {$v_2$};
    \node (C) at (2,-2) {$v_3$};
    \node (D) at (5,2) {$v_4$};
    \node (E) at (-2,-2) {$v_5$};
\end{scope}

\begin{scope}[>={Stealth[black]},
              every node/.style={fill=white,circle},
              every edge/.style={draw=red,very thick}]
    \path [->] (A) edge (E);
    \path [->] (E) edge (C);
    \path [->] (C) edge (B);
    \path [->] (B) edge (A);
\end{scope}
\end{tikzpicture}

\section*{Задание № 2.}

Дан ориентированный граф $D$.

1. Найдите матрицу смежности $A$.

2. С помощью алгоритма фронта волны найдите расстояния в
ориентированном графе $D$, диаметр, радиус и центры.

\medskip

\begin{tikzpicture}
\begin{scope}[every node/.style={circle,thick,draw}]
    \node (A) at (-2.5,1) {$v_1$};
    \node (B) at (0,3) {$v_2$};
    \node (C) at (2.5,1) {$v_3$};
    \node (D) at (1.8,-2.2) {$v_4$};
    \node (E) at (-1.8,-2.2) {$v_5$};
    \node (F) at (0,0) {$v_6$};
\end{scope}

\begin{scope}[>={Stealth[black]},
              every node/.style={fill=white,circle},
              every edge/.style={draw=red,very thick}]
    \path [->] (A) edge (F);
    \path [->] (B) edge (A);
    \path [->] (B) edge (C);
    \path [->] (B) edge (F);
    \path [->] (C) edge (D);
    \path [->] (D) edge (B);
    \path [->] (D) edge (F);
    \path [->] (E) edge (D);
    \path [->] (F) edge (C);
    \path [->] (F) edge (E);
    \path [->] (A) edge[bend left=10] (E);
    \path [->] (E) edge[bend left=10] (A);
\end{scope}
\end{tikzpicture}

\begin{center}Решение:\end{center}

1) Матрица смежности ориентированного графа $D$ имеет вид:

$$A(D)=
\begin{pmatrix}
0 & 0 & 0 & 0 & 1 & 1 \\
1 & 0 & 1 & 0 & 0 & 1 \\
0 & 0 & 0 & 1 & 0 & 0 \\
0 & 1 & 0 & 0 & 0 & 1 \\
1 & 0 & 0 & 1 & 0 & 0 \\
0 & 0 & 1 & 0 & 1 & 0 \\
\end{pmatrix}
$$

2) Заполним матрицу минимальных расстояний, сперва поставив нули по главной диагонали, затем перенеся единицы из матрицы смежности, и после этого используя алгоритм фронта волны для каждой из оставшихся пар вершин графа $D$. А именно: фиксикуем строку, смотрим в какие неотмеченные вершины мы можем попасть из единичек в данной строке за один шаг и заполняем их двойками. Так заполняем двойками все строки. Далее смотрим в какие неотмеченные вершины мы можем попасть за один шаг из вершин с двойками в этой строке и заполняем их тройками. На этом шаге матрица минимальных расстояний у нас заполнится.

$$R(D)=
\begin{pmatrix}
0 & 3 & 2 & 2 & 1 & 1 \\
1 & 0 & 1 & 2 & 2 & 1 \\
3 & 2 & 0 & 1 & 3 & 2 \\
2 & 1 & 2 & 0 & 2 & 1 \\
1 & 2 & 3 & 1 & 0 & 2 \\
2 & 3 & 1 & 2 & 1 & 0 \\
\end{pmatrix}
$$

Из матрицы $R(D)$ минимальных расстояний находим диаметр:

$$d(D)=\max\limits_{v,w\in V} d(v,w)=r_{12}=3.$$

Для каждой вершины ориентированного графа $D$ найдем эксцентриситет (максимальное удаление от вершины) по формуле $r(v)=\max\limits_{w\in V} d(v,w)$:

$$r(v_1)=3,\quad r(v_2)=2,\quad r(v_3)=3,\quad r(v_4)=2,\quad r(v_5)=3,\quad r(v_6)=3.$$

Значит радиусом графа $D$ будет:

$$r(D)=\min\limits_{v\in V}r(v)=2.$$

Центрами графа будут являться две вершины $v_2$ и $v_4$, так как обе они имеют минимальный эксцентриситет.

\section*{Задание № 3.}

Найдите минимальный путь в нагруженном ориентированном графе $D$ по методу Форда-Беллмана из указанной вершины в указанную.

Из $v_5$ в $v_3$.

\medskip

\begin{tikzpicture}
\begin{scope}[every node/.style={circle,thick,draw}]
    \node (A) at (-4,0) {$v_1$};
    \node (B) at (-2,3) {$v_2$};
    \node (C) at (2,3) {$v_3$};
    \node (D) at (4,0) {$v_4$};
    \node (E) at (2,-3) {$v_5$};
    \node (F) at (-2,-3) {$v_6$};
    \node (G) at (0,0) {$v_7$};
\end{scope}

\begin{scope}[>={Stealth[black]},
              every node/.style={fill=white,circle},
              every edge/.style={draw=red,very thick}]
    \path [->] (A) edge node {$6$} (F);
    \path [->] (A) edge node[pos=0.75] {$8$} (G);
    \path [->] (B) edge node {$7$} (A);
    \path [->] (B) edge node {$3$} (C);
    \path [->] (B) edge node[pos=0.25] {$9$} (D);
    \path [->] (C) edge node {$11$} (D);
    \path [->] (C) edge node[pos=0.2] {$8$} (E);
    \path [->] (D) edge node[pos=0.75] {$12$} (F);
    \path [->] (D) edge node[pos=0.25] {$8$} (G);
    \path [->] (E) edge node {$4$} (D);
    \path [->] (E) edge node {$9$} (F);
    \path [->] (F) edge node[pos=0.75] {$5$} (B);
    \path [->] (F) edge node {$7$} (G);
    \path [->] (G) edge node {$2$} (B);
    \path [->] (G) edge node[pos=0.25] {$10$} (C);
    \path [->] (G) edge node[pos=0.25] {$7$} (E);
\end{scope}
\end{tikzpicture}

\begin{center}Решение:\end{center}

Матрица длин дуг для нагруженного ориентированного графа $D$ будет следующей:
$$C(D)=
\begin{pmatrix}
\infty & \infty & \infty & \infty & \infty & 6 & 8 \\
7 & \infty & 3 & 9 & \infty & \infty & \infty \\
\infty & \infty & \infty & 11 & 8 & \infty & \infty \\
\infty & \infty & \infty & \infty & \infty & 12 & 8 \\
\infty & \infty & \infty & 4 & \infty & 9 & \infty \\
\infty & 5 & \infty & \infty & \infty & \infty & 7 \\
\infty & 2 & 10 & \infty & 7 & \infty & \infty \\
\end{pmatrix}
$$

Согласно алгоритму метода Форда-Беллмана, составляем таблицу стоимости минимальных путей из вершины $v_5$ в вершину $v_i$ не более, чем за $k$ шагов. Заполняем пятую строку нулями, а элементы первого столбца, кроме пятого, заполняем $\infty$. Второй столбец, кроме пятого элемента, заполняем пятой строкой матрицы длин дуг $C(D)$. Далее каждый следующий столбец получаем находя минимум чисел полученых сложением соответствующих элементов предыдущего столбца таблицы с элементами соответствующего столбца матрицы длин дуг $C(D)$.

\medskip
\bgroup
\def\arraystretch{1.5}
\setlength\tabcolsep{6}
\begin{tabular}{|>{\columncolor{Gray}}c|c|c|c|c|c|c|c|}
\hline
\rowcolor{Gray}
\cellcolor{white} & $\lambda_i^{(0)}$ & $\lambda_i^{(1)}$ & $\lambda_i^{(2)}$ & $\lambda_i^{(3)}$ & $\lambda_i^{(4)}$ & $\lambda_i^{(5)}$ & $\lambda_i^{(6)}$ \\
\hline
$v_1$ & $\infty$ & $\infty$ & $\infty$ & 21 & 21 & 21 & 21 \\
\hline
$v_2$ & $\infty$ & $\infty$ & 14 & 14 & 14 & 14 & 14 \\
\hline
$v_3$ & $\infty$ & $\infty$ & $\infty$ & 17 & 17 & 17 & 17 \\
\hline
$v_4$ & $\infty$ & 4 & 4 & 4 & 4 & 4 & 4 \\
\hline
$v_5$ & 0 & 0 & 0 & 0 & 0 & 0 & 0 \\
\hline
$v_6$ & $\infty$ & 9 & 9 & 9 & 9 & 9 & 9 \\
\hline
$v_7$ & $\infty$ & $\infty$ & 12 & 12 & 12 & 12 & 12 \\
\hline
\end{tabular}
\egroup
\medskip

$\lambda_1^{(2)} =\min\{\infty+\infty,\; \infty+7,\; \infty+\infty,\; 4+\infty,\; 0+\infty,\; 9+\infty,\; \infty+\infty\}=\infty.$

$\lambda_2^{(2)} =\min\{\infty+\infty,\; \infty+\infty,\; \infty+\infty,\; 4+\infty,\; 0+\infty,\; 9+5,\; \infty+2\}=14.$

$\lambda_3^{(2)} =\min\{\infty+\infty,\; \infty+3,\; \infty+\infty,\; 4+\infty,\; 0+\infty,\; 9+\infty,\; \infty+10\}=\infty.$

$\lambda_4^{(2)} =\min\{\infty+\infty,\; \infty+9,\; \infty+11,\; 4+\infty,\; 0+4,\; 9+\infty,\; \infty+\infty\}=4.$

$\lambda_6^{(2)} =\min\{\infty+6,\; \infty+\infty,\; \infty+\infty,\; 4+12,\; 0+9,\; 9+\infty,\; \infty+\infty\}=9.$

$\lambda_7^{(2)} =\min\{\infty+8,\; \infty+\infty,\; \infty+\infty,\; 4+8,\; 0+\infty,\; 9+7,\; \infty+\infty\}=12.$

\medskip

$\lambda_1^{(3)} =\min\{\infty+\infty,\; 14+7,\; \infty+\infty,\; 4+\infty,\; 0+\infty,\; 9+\infty,\; 12+\infty\}=21.$

$\lambda_2^{(3)} =\min\{\infty+\infty,\; 14+\infty,\; \infty+\infty,\; 4+\infty,\; 0+\infty,\; 9+5,\; 12+2\}=14.$

$\lambda_3^{(3)} =\min\{\infty+\infty,\; 14+3,\; \infty+\infty,\; 4+\infty,\; 0+\infty,\; 9+\infty,\; 12+10\}=17.$

$\lambda_4^{(3)} =\min\{\infty+\infty,\; 14+9,\; \infty+11,\; 4+\infty,\; 0+4,\; 9+\infty,\; 12+\infty\}=4.$

$\lambda_6^{(3)} =\min\{\infty+6,\; 14+\infty,\; \infty+\infty,\; 4+12,\; 0+9,\; 9+\infty,\; 12+\infty\}=9.$

$\lambda_7^{(3)} =\min\{\infty+8,\; 14+\infty,\; \infty+\infty,\; 4+8,\; 0+\infty,\; 9+7,\; 12+\infty\}=12.$

\medskip

$\lambda_1^{(4)} =\min\{21+\infty,\; 14+7,\; 17+\infty,\; 4+\infty,\; 0+\infty,\; 9+\infty,\; 12+\infty\}=21.$

$\lambda_2^{(4)} =\min\{21+\infty,\; 14+\infty,\; 17+\infty,\; 4+\infty,\; 0+\infty,\; 9+5,\; 12+2\}=14.$

$\lambda_3^{(4)} =\min\{21+\infty,\; 14+3,\; 17+\infty,\; 4+\infty,\; 0+\infty,\; 9+\infty,\; 12+10\}=17.$

$\lambda_4^{(4)} =\min\{21+\infty,\; 14+9,\; 17+11,\; 4+\infty,\; 0+4,\; 9+\infty,\; 12+\infty\}=4.$

$\lambda_6^{(4)} =\min\{21+6,\; 14+\infty,\; 17+\infty,\; 4+12,\; 0+9,\; 9+\infty,\; 12+\infty\}=9.$

$\lambda_7^{(4)} =\min\{21+8,\; 14+\infty,\; 17+\infty,\; 4+8,\; 0+\infty,\; 9+7,\; 12+\infty\}=12.$

\medskip

Следующие два столбца $\lambda_i^{(5)}$ и $\lambda_i^{(6)}$ вычисляются полностью идентично предыдущему.

Теперь необходимо по построенной таблице и по матрице $C(D)$ восстановить минимальный путь из вершины $v_5$ в $v_3$, который будет строиться с конца, то есть начиная с вершины $v_3$.

Для этого выбираем минимальное из чисел, стоящих в строке, соответствующей $v_3$ в таблице. Это число – $17$ является длиной минимального искомого пути. В первый раз такая длина была получена при количестве шагов, равном $3$.

В вершину $v_3$ мы можем попасть за один шаг из вершин $v_2$ и $v_7$ (длина соответствующих дуг $3$ и $10$ соответственно – данные из матрицы $C(D)$). Выбираем минимальную по длине из этих дуг. Далее, в вершину $v_2$ можно за два хода попасть только через вершину $v_6$.

Таким образом искомый путь за $3$ шага минимальной длины из вершины $v_5$ в $v_3$ найден: $\langle v_5, v_6, v_2, v_3\rangle$.

\section*{Задание № 4.}

Найдите минимальное остовное дерево в неориентированном нагруженном графе.

\medskip

\begin{tikzpicture}
\begin{scope}[every node/.style={circle,thick,draw}]
    \node (A) at (-4,0) {$v_1$};
    \node (B) at (-2,3) {$v_2$};
    \node (C) at (2,3) {$v_3$};
    \node (D) at (4,0) {$v_4$};
    \node (E) at (2,-3) {$v_5$};
    \node (F) at (-2,-3) {$v_6$};
    \node (G) at (0,0) {$v_7$};
\end{scope}

\begin{scope}[-,
              every node/.style={fill=white,circle},
              every edge/.style={draw=red,very thick}]
    \path (A) edge node {$2$} (B);
    \path (A) edge node[pos=0.25] {$12$} (C);
    \path (A) edge node[pos=0.25] {$10$} (E);
    \path (A) edge node {$8$} (F);
    \path (A) edge node {$1$} (G);
    \path (G) edge node[pos=0.25] {$5$} (B);
    \path (G) edge node[pos=0.25] {$8$} (C);
    \path (G) edge node[pos=0.25] {$5$} (D);
    \path (G) edge node[pos=0.25] {$7$} (E);
    \path (G) edge node[pos=0.25] {$3$} (F);
    \path (B) edge node[pos=0.2] {$3$} (D);
    \path (D) edge node[pos=0.2] {$4$} (F);
    \path (C) edge node[pos=0.2] {$9$} (E);
    \path (B) edge node {$6$} (C);
    \path (C) edge node {$7$} (D);
    \path (D) edge node {$12$} (E);
    \path (E) edge node {$12$} (F);
\end{scope}
\end{tikzpicture}

\begin{center}Решение:\end{center}

$$
C(G)=
\begin{pmatrix}
0 & 2 & 12 & \infty & 10 & 8 & 1 \\
2 & 0 & 6 & 3 & \infty & \infty & 5 \\
12 & 6 & 0 & 7 & 9 & \infty & 8 \\
\infty & 3 & 7 & 0 & 12 & 4 & 5 \\
10 & \infty & 9 & 12 & 0 & 12 & 7 \\
8 & \infty & \infty & 4 & 12 & 0 & 3 \\
1 & 5 & 8 & 5 & 7 & 3 & 0 \\
\end{pmatrix}
$$

$G_2=(\{v_1,v_7\},\{x_{17}\})$,\quad $l(G_2)=c_{17}=1$;

$G_3=(\{v_2,v_1,v_7\},\{x_{21},x_{17}\})$,\quad $l(G_3)=l(G_2)+c_{21}=1+2=3$;

$G_4=(\{v_4,v_2,v_1,v_7\},\{x_{42},x_{21},x_{17}\})$,\quad $l(G_4)=l(G_3)+c_{42}=3+3=6$;

$G_5=(\{v_4,v_2,v_1,v_7,v_6\},\{x_{42},x_{21},x_{17},x_{76}\})$,\quad $l(G_5)=l(G_4)+c_{76}=6+3=9$;

$G_6=(\{v_3,v_4,v_2,v_1,v_7,v_6\},\{x_{23},x_{42},x_{21},x_{17},x_{76}\})$,\quad $l(G_6)=l(G_5)+c_{23}=9+6=15$;

$G_7=(\{v_3,v_4,v_2,v_1,v_7,v_6,v_5\},\{x_{23},x_{42},x_{21},x_{17},x_{76},x_{75}\})$,\quad $l(G_7)=l(G_6)+c_{75}=15+7=22$;

Поскольку $i=7=n(G)$, то работа алгоритма на этом заканчивается.

Таким образом, искомое минимальное остовное дерево графа $G$ --− это граф $G_7$, изображенный ниже, длина которого равна $22$.

\medskip

\begin{tikzpicture}
\begin{scope}[every node/.style={circle,thick,draw}]
    \node (A) at (-4,0) {$v_1$};
    \node (B) at (-2,3) {$v_2$};
    \node (C) at (2,3) {$v_3$};
    \node (D) at (4,0) {$v_4$};
    \node (E) at (2,-3) {$v_5$};
    \node (F) at (-2,-3) {$v_6$};
    \node (G) at (0,0) {$v_7$};
\end{scope}

\begin{scope}[-,
              every node/.style={fill=white,circle},
              every edge/.style={draw=red,very thick}]
    \path (B) edge node {$6$} (C);
    \path (D) edge node {$3$} (B);
    \path (B) edge node {$2$} (A);
    \path (A) edge node {$1$} (G);
    \path (G) edge node {$3$} (F);
    \path (G) edge node {$7$} (E);
\end{scope}
\end{tikzpicture}

\section*{Задание № 5.}

Методом ветвей и границ найдите оптимальный путь коммивояжёра при следующей матрице стоимости.

\medskip

\begin{tabular}{|>{\columncolor{Gray}}c|c|c|c|c|c|}
\hline
\rowcolor{Gray}
\cellcolor{white} & 1 & 2 & 3 & 4 & 5 \\
\hline
1 & $\infty$ & 4 & 5 & 9 & 10 \\
\hline
2 & 4 & $\infty$ & 3 & 10 & 11 \\
\hline
3 & 5 & 3 & $\infty$ & 9 & 8 \\
\hline
4 & 9 & 10 & 9 & $\infty$ & 5 \\
\hline
5 & 10 & 11 & 8 & 5 & $\infty$ \\
\hline
\end{tabular}

\begin{center}Решение:\end{center}

Приведём матрицу по строкам:

\medskip
\begin{tabular}{|>{\columncolor{Gray}}c|c|c|c|c|c|}
\hline
\rowcolor{Gray}
\cellcolor{white} & 1 & 2 & 3 & 4 & 5 \\
\hline
1 & $\infty$ & 0 & 1 & 5 & 6 \\
\hline
2 & 1 & $\infty$ & 0 & 7 & 8 \\
\hline
3 & 2 & 0 & $\infty$ & 6 & 5 \\
\hline
4 & 4 & 5 & 4 & $\infty$ & 0 \\
\hline
5 & 5 & 6 & 3 & 0 & $\infty$ \\
\hline
\end{tabular}
\medskip

Затем по столбцам:

\medskip
\begin{tabular}{|>{\columncolor{Gray}}c|c|c|c|c|c|}
\hline
\rowcolor{Gray}
\cellcolor{white} & 1 & 2 & 3 & 4 & 5 \\
\hline
1 & $\infty$ & 0 & 1 & 5 & 6 \\
\hline
2 & 0 & $\infty$ & 0 & 7 & 8 \\
\hline
3 & 1 & 0 & $\infty$ & 6 & 5 \\
\hline
4 & 3 & 5 & 4 & $\infty$ & 0 \\
\hline
5 & 4 & 6 & 3 & 0 & $\infty$ \\
\hline
\end{tabular}
\medskip

Обозначим полученную выше матрицу за $C_1$.

Сумма констант приведения $\varphi\left(\Gamma\right)=4+3+3+5+5+1=21$.

Найдём веса нулей в матрице $C_1$:

\medskip
\begin{tabular}{|>{\columncolor{Gray}}c|c|c|c|c|c|}
\hline
\rowcolor{Gray}
\cellcolor{white} & 1 & 2 & 3 & 4 & 5 \\
\hline
1 & $\infty$ & $0_{(1)}$ & 1 & 5 & 6 \\
\hline
2 & $0_{(1)}$ & $\infty$ & $0_{(1)}$ & 7 & 8 \\
\hline
3 & 1 & $0_{(1)}$ & $\infty$ & 6 & 5 \\
\hline
4 & 3 & 5 & 4 & $\infty$ & $0_{(3)}$ \\
\hline
5 & 4 & 6 & 3 & $0_{(3)}$ & $\infty$ \\
\hline
\end{tabular}
\medskip

Самыми тяжёлыми оказались нули в клетках $(4,5)$ и $(5,4)$. Мы можем выбрать любую из них, пусть это будет клетка $(5,4)$.

Матрицу $C_{1,1}$, которая соответстует множеству $\Gamma_{\{(5,4)\}}$, получим вычеркиванием четвёртого столбца и пятой строки из матрицы $C_1$, причём число в клетке $(4,5)$ нужно заменить на $\infty$, чтобы не получалось коротких циклов:

\medskip
\begin{tabular}{|>{\columncolor{Gray}}c|c|c|c|c|}
\hline
\rowcolor{Gray}
\cellcolor{white} & 1 & 2 & 3 & 5 \\
\hline
1 & $\infty$ & 0 & 1 & 6 \\
\hline
2 & 0 & $\infty$ & 0 & 8 \\
\hline
3 & 1 & 0 & $\infty$ & 5 \\
\hline
4 & 3 & 5 & 4 & $\infty$ \\
\hline
\end{tabular}
\medskip

Приведем матрицу по строкам:

\medskip
\begin{tabular}{|>{\columncolor{Gray}}c|c|c|c|c|}
\hline
\rowcolor{Gray}
\cellcolor{white} & 1 & 2 & 3 & 5 \\
\hline
1 & $\infty$ & 0 & 1 & 6 \\
\hline
2 & 0 & $\infty$ & 0 & 8 \\
\hline
3 & 1 & 0 & $\infty$ & 5 \\
\hline
4 & 0 & 2 & 1 & $\infty$ \\
\hline
\end{tabular}
\medskip

После приведения по столбцам матрица $C_{1,1}$ будет иметь следующий вид:

\medskip
\begin{tabular}{|>{\columncolor{Gray}}c|c|c|c|c|}
\hline
\rowcolor{Gray}
\cellcolor{white} & 1 & 2 & 3 & 5 \\
\hline
1 & $\infty$ & 0 & 1 & 1 \\
\hline
2 & 0 & $\infty$ & 0 & 3 \\
\hline
3 & 1 & 0 & $\infty$ & 0 \\
\hline
4 & 0 & 2 & 1 & $\infty$ \\
\hline
\end{tabular}
\medskip

$\varphi\left(\Gamma_{\{(5,4)\}}\right)=\varphi\left(\Gamma\right)+3+5=21+3+5=29$.

Множеству же $\Gamma_{\{\overline{(5,4)}\}}$, соответствует матрица $C_{1,2}$, полученная заменой на $\infty$ элемента $(5,4)$ в матрице $C_1$:

\medskip
\begin{tabular}{|>{\columncolor{Gray}}c|c|c|c|c|c|}
\hline
\rowcolor{Gray}
\cellcolor{white} & 1 & 2 & 3 & 4 & 5 \\
\hline
1 & $\infty$ & 0 & 1 & 5 & 6 \\
\hline
2 & 0 & $\infty$ & 0 & 7 & 8 \\
\hline
3 & 1 & 0 & $\infty$ & 6 & 5 \\
\hline
4 & 3 & 5 & 4 & $\infty$ & 0 \\
\hline
5 & 4 & 6 & 3 & $\infty$ & $\infty$ \\
\hline
\end{tabular}
\medskip

Матрица $C_{1,2}$ после приведения по строкам:

\medskip
\begin{tabular}{|>{\columncolor{Gray}}c|c|c|c|c|c|}
\hline
\rowcolor{Gray}
\cellcolor{white} & 1 & 2 & 3 & 4 & 5 \\
\hline
1 & $\infty$ & 0 & 1 & 5 & 6 \\
\hline
2 & 0 & $\infty$ & 0 & 7 & 8 \\
\hline
3 & 1 & 0 & $\infty$ & 6 & 5 \\
\hline
4 & 3 & 5 & 4 & $\infty$ & 0 \\
\hline
5 & 1 & 3 & 0 & $\infty$ & $\infty$ \\
\hline
\end{tabular}
\medskip

Затем по столбцам:

\medskip
\begin{tabular}{|>{\columncolor{Gray}}c|c|c|c|c|c|}
\hline
\rowcolor{Gray}
\cellcolor{white} & 1 & 2 & 3 & 4 & 5 \\
\hline
1 & $\infty$ & 0 & 1 & 0 & 6 \\
\hline
2 & 0 & $\infty$ & 0 & 2 & 8 \\
\hline
3 & 1 & 0 & $\infty$ & 1 & 5 \\
\hline
4 & 3 & 5 & 4 & $\infty$ & 0 \\
\hline
5 & 1 & 3 & 0 & $\infty$ & $\infty$ \\
\hline
\end{tabular}
\medskip

$\varphi\left(\Gamma_{\{\overline{(5,4)}\}}\right)=21+3+5=29$

В нашем случае $\varphi\left(\Gamma_{\{\overline{(5,4)}\}}\right)=\varphi\left(\Gamma_{\{(5,4)\}}\right)=29$, значит для дальнейшей обработки мы можем выбрать любое из множеств $\Gamma_{\{(5,4)\}}$ или $\Gamma_{\{\overline{(5,4)}\}}$, пусть это будет множество $\Gamma_{\{(5,4)\}}$.

Посчитаем веса нулей в матрице $C_{1,1}$:

\medskip
\begin{tabular}{|>{\columncolor{Gray}}c|c|c|c|c|}
\hline
\rowcolor{Gray}
\cellcolor{white} & 1 & 2 & 3 & 5 \\
\hline
1 & $\infty$ & $0_{(1)}$ & 1 & 1 \\
\hline
2 & $0_{(0)}$ & $\infty$ & $0_{(1)}$ & 3 \\
\hline
3 & 1 & $0_{(0)}$ & $\infty$ & $0_{(1)}$ \\
\hline
4 & $0_{(1)}$ & 2 & 1 & $\infty$ \\
\hline
\end{tabular}
\medskip

Самыми тяжёлыми являются нули с номерами $(1,2)$, $(2,3)$, $(3,5)$, $(4,1)$, выберем любую из этих клеток, пусть это будет клетка $(4,1)$, так что теперь следует рассматривать множества $\Gamma_{\{(5,4);(4,1)\}}$ и $\Gamma_{\{(5,4);\overline{(4,1)}\}}$.

Матрицу $C_{1,1,1}$ получим вычеркиванием первого столбца и четвёртой строки из матрицы $C_{1,1}$, причём число в клетке $(1,5)$ нужно будет заменить на $\infty$:

\medskip
\begin{tabular}{|>{\columncolor{Gray}}c|c|c|c|}
\hline
\rowcolor{Gray}
\cellcolor{white} & 2 & 3 & 5 \\
\hline
1 & 0 & 1 & $\infty$ \\
\hline
2 & $\infty$ & 0 & 3 \\
\hline
3 & 0 & $\infty$ & 0 \\
\hline
\end{tabular}
\medskip

Она сразу получилась приведённой так что $\varphi\left(\Gamma_{\{(5,4);(4,1)\}}\right)=29$.

Матрицу же $C_{1,1,2}$, получим заменой на $\infty$ элемента $(4,1)$ в матрице $C_{1,1}$:

\medskip
\begin{tabular}{|>{\columncolor{Gray}}c|c|c|c|c|}
\hline
\rowcolor{Gray}
\cellcolor{white} & 1 & 2 & 3 & 5 \\
\hline
1 & $\infty$ & 0 & 1 & 1 \\
\hline
2 & 0 & $\infty$ & 0 & 3 \\
\hline
3 & 1 & 0 & $\infty$ & 0 \\
\hline
4 & $\infty$ & 2 & 1 & $\infty$ \\
\hline
\end{tabular}
\medskip

После приведения она будет иметь вид:

\medskip
\begin{tabular}{|>{\columncolor{Gray}}c|c|c|c|c|}
\hline
\rowcolor{Gray}
\cellcolor{white} & 1 & 2 & 3 & 5 \\
\hline
1 & $\infty$ & 0 & 1 & 1 \\
\hline
2 & 0 & $\infty$ & 0 & 3 \\
\hline
3 & 1 & 0 & $\infty$ & 0 \\
\hline
4 & $\infty$ & 1 & 0 & $\infty$ \\
\hline
\end{tabular}
\medskip

$\varphi\left(\Gamma_{\{(5,4);\overline{(4,1)}\}}\right)=29+1=30$.

В этом случае сумма констант приведения больше чем в предыдущем,
следовательно дальнейшей обработке подлежит множество $\Gamma_{\{(5,4);(4,1)\}}$.


Найдём веса нулей матрицы $C_{1,1,1}$:

\medskip
\begin{tabular}{|>{\columncolor{Gray}}c|c|c|c|}
\hline
\rowcolor{Gray}
\cellcolor{white} & 2 & 3 & 5 \\
\hline
1 & $0_{(1)}$ & 1 & $\infty$ \\
\hline
2 & $\infty$ & $0_{(1)}$ & 3 \\
\hline
3 & $0_{(0)}$ & $\infty$ & $0_{(3)}$ \\
\hline
\end{tabular}
\medskip

Самым тяжёлым является нуль с номером $(3,5)$, так что теперь следует рассматривать множества $\Gamma_{\{(5,4);(4,1);(3,5)\}}$ и $\Gamma_{\{(5,4);(4,1);\overline{(3,5)}\}}$.

Матрицу $C_{1,1,1,1}$ получим вычеркиванием пятого столбца и третьей строки из матрицы $C_{1,1,1}$, причём число в клетке $(1,3)$ нужно будет заменить на $\infty$:

\medskip
\begin{tabular}{|>{\columncolor{Gray}}c|c|c|}
\hline
\rowcolor{Gray}
\cellcolor{white} & 2 & 3 \\
\hline
1 & 0 & $\infty$ \\
\hline
2 & $\infty$ & 0 \\
\hline
\end{tabular}
\medskip

Она у нас сразу приведённая и $\varphi\left(\Gamma_{\{(5,4);(4,1);(3,5)\}}\right)=29$.

Раз сумма констант последнего приведения равна нулю, то ветвь соответствующую множеству $\Gamma_{\{(5,4);(4,1);\overline{(3,5)}\}}$ можно не рассматривать.

Нулевые клетки матрицы $C_{1,1,1,1}$ дают те дуги, которые с найденными ранее и составляют обход коммивояжёра, причём вес этого обхода равен значению оценочной функции, т.е. 29. Вот этот обход:

$(1,2) (2,3) (3,5) (5,4) (4,1)$ или
$1\to2\to3\to5\to4\to1$.

\end{document}
