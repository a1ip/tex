\documentclass[fleqn]{article}
\usepackage{fullpage}
%%% Работа с русским языком
\usepackage[T2A]{fontenc}
\usepackage[utf8]{inputenc}
\usepackage[english,russian]{babel}   %% загружает пакет многоязыковой вёрстки
\usepackage{indentfirst}
\frenchspacing
\usepackage{titlesec} % package to customize chapters, sections and subsections style
%--------------------------------------
\titleformat{\section}{\large\bfseries\centering}{\thesection}{1em}{}
%Hyphenation rules
%--------------------------------------
\usepackage{hyphenat}
\hyphenation{мате-мати-ка восста-навливать}
%--------------------------------------
\usepackage[intlimits,sumlimits,fleqn]{amsmath}
\usepackage{amsfonts,amssymb,amsthm,mathtools}
\usepackage[bb=pazo,frak=pxtx,frakscaled=1.3]{mathalfa}
\usepackage{upgreek}
\usepackage{graphicx}
\usepackage{verbatim} %% для многострочных комментариев

%% Перенос знаков в формулах (по Львовскому)
\newcommand*{\hm}[1]{#1\nobreak\discretionary{}
{\hbox{$\mathsurround=0pt #1$}}{}}

\renewcommand{\le}{\ensuremath{\leqslant}}
\renewcommand{\leq}{\ensuremath{\leqslant}}
\renewcommand{\ge}{\ensuremath{\geqslant}}
\renewcommand{\geq}{\ensuremath{\geqslant}}

\newcommand{\tg}{\mathop{\rm tg}\nolimits}

\usepackage{pgfplots}
\pgfplotsset{width=10cm,compat=1.9}
\usepgfplotslibrary{external}
\tikzexternalize
\tikzset{>=latex}

\title{Задачка по алгебре}
\author{Ригованов Филипп Юрьевич}
\begin{document}
\date{}
\maketitle
\section*{Условие}
Моторная лодка проплывает по реке 2 км против течения за тоже время, что она проплывает 3 км по течению. Расстояние между двумя пристанями 24 км. Для того, чтобы преодолеть это расстояние туда и обратно моторной лодке требуется 2,5 часа. Найти собственную скорость лодки(скорость в стоячей воде) и скорость течения.

Решение:

Пусть $x$ км/ч - скрость лодки против течения, а $y$ км/ч - скорость лодки по течению. Тогда имеем систему уравнений:


$$\begin{cases}
\frac{2}{x}=\frac{3}{y}\\
\frac{24}{x}+\frac{24}{y}=2,5
\end{cases}
\begin{cases}
2y=3x\\
\frac{24}{x}+\frac{24}{y}=2,5
\end{cases}
\begin{cases}
y=1,5x\\
\frac{24}{x}+\frac{24}{1,5x}=2,5
\end{cases}
\begin{cases}
y=1,5x\\
\frac{24\cdot3}{3x}+\frac{24\cdot2}{3x}=2,5
\end{cases}
\begin{cases}
y=1,5x\\
24\cdot5=2,5\cdot3x
\end{cases}
\begin{cases}
y=1,5x\\
x=\frac{24\cdot5}{3\cdot2,5}
\end{cases}
$$

$$
\begin{cases}
y=1,5x\\
x=16
\end{cases}
\begin{cases}
y=24\\
x=16
\end{cases}
$$

Пусть теперь $u$ км/ч - скорость течения, а $v$ км/ч - собственная скорость лодки.

Тогда $u=\frac{y-x}{2}=\frac{8}{2}=4$ км/ч, $v=x+u=16+4=20$ км/ч.

Ответ: 20 км/ч и 4 км/ч.

\end{document}
