\documentclass{article}
\usepackage{fullpage}
%%% Работа с русским языком
\usepackage[T2A]{fontenc}
\usepackage[utf8]{inputenc}
\usepackage[english,russian]{babel}   %% загружает пакет многоязыковой вёрстки
%\usepackage{fontspec}      %% подготавливает загрузку шрифтов Open Type, True Type и др.
%\defaultfontfeatures{Ligatures={TeX},Renderer=Basic}  %% свойства шрифтов по умолчанию
%\setmainfont[Ligatures={TeX,Historic}]{Times New Roman} %% задаёт основной шрифт документа
%\setsansfont{Comic Sans MS}                    %% задаёт шрифт без засечек
%\setmonofont{Courier New}
\usepackage{indentfirst}
\frenchspacing
\usepackage{titlesec} % package to customize chapters, sections and subsections style
%--------------------------------------
\titleformat{\section}{\large\bfseries\centering}{\thesection}{1em}{}
%Hyphenation rules
%--------------------------------------
\usepackage{hyphenat}
\hyphenation{мате-мати-ка восста-навливать}
%--------------------------------------
\usepackage{xcolor}
\usepackage{pmboxdraw}
\usepackage{listingsutf8}
\lstset{
    language=[WinXP]{command.com},
    basicstyle=\ttfamily\small,
    numberstyle=\footnotesize,
    numbers=left,
    backgroundcolor=\color{gray!10},
    frame=single,
    tabsize=2,
    rulecolor=\color{black!30},
    title=\lstname,
    escapeinside={\%*}{*)},
    breaklines=true,
    breakatwhitespace=true,
    framextopmargin=2pt,
    framexbottommargin=2pt,
    extendedchars=false,
    inputencoding=utf8
}

\title{Лабораторная работа № 2 по курсу операционные системы \\
Тема: «Операционная система MS DOS и Windows: \\
конфигурирование и настройка»}
\author{Ригованов Филипп Юрьевич, студент группы КТбз1-1}
\date{Декабрь 2015}

\begin{document}

\maketitle
\section*{Часть № 1. MS DOS.}
\begin{enumerate}
\item
Просмотрим содержимое диска С:
\begin{lstlisting}[language={},numbers={}]
C:\Documents and Settings\phil>dir C:\
 %*Том в устройстве C не имеет метки.*)
 %*Серийный номер тома*): BC51-556D

 %*Содержимое папки*) C:\

26.03.2015  11:43                 0 AUTOEXEC.BAT
26.03.2015  11:58    <DIR>          ce428f8bace5b7b1331098b2640510
26.03.2015  11:43                 0 CONFIG.SYS
26.03.2015  11:45    <DIR>          Documents and Settings
17.12.2015  23:19    <DIR>          PABCWork.NET
17.12.2015  14:53    <DIR>          Program Files
03.06.2015  20:34    <DIR>          WINDOWS
               2 %*файлов*)              0 %*байт*)
               5 %*папок*)  11 720 065 024 %*байт свободно*)

C:\Documents and Settings\phil>
\end{lstlisting}
В корневом каталоге 5 директорий и 2 файла.
\newpage
\item
Создаём каталог WORK с вложенными папками, переходим в него и выводим на экран дерево каталогов.
\begin{lstlisting}[language={},numbers={}]
C:\Documents and Settings\phil>mkdir C:\WORK

C:\Documents and Settings\phil>cd C:\WORK

C:\WORK>mkdir PART1 PART2 PART3\FIRST PART3\SECOND

C:\WORK>tree C:\WORK
%*Структура папок*)
%*Серийный номер тома*): BC51-556D
C:\WORK
%*├───*)PART1
%*├───*)PART2
%*└───*)PART3
    %*├───*)FIRST
    %*└───*)SECOND

C:\WORK>
\end{lstlisting}
\item
Создаём файл inform.txt с текстом <<Задание практической работы №1>>:
\begin{lstlisting}[language={},numbers={}]
C:\WORK>copy /a con PART3\FIRST\inform.txt
%*Задание практической работы №1*)^Z
%*Скопировано файлов*):         1.

C:\WORK>
\end{lstlisting}
\item
Копируем файл inform.txt в каталог SECOND и дописываем в него свою фамилию:
\begin{lstlisting}[language={},numbers={}]
C:\WORK>copy PART3\FIRST\inform.txt PART3\SECOND
%*Скопировано файлов*):         1.

C:\WORK>echo %*Ригованов*) >> PART3\SECOND\inform.txt

C:\WORK>
\end{lstlisting}
\item
Скопированный файл inform.txt переименовываем в new\_inf.txt:
\begin{lstlisting}[language={},numbers={}]
C:\WORK>ren PART3\SECOND\inform.txt new_inf.txt

C:\WORK>
\end{lstlisting}
\newpage
\item
Копируем new\_inf.txt в каталог PART1 и удаляем каталог PART3:
\begin{lstlisting}[language={},numbers={}]
C:\WORK>copy PART3\SECOND\new_inf.txt PART1
%*Скопировано файлов*):         1.

C:\WORK>rd PART3 /s/q

C:\WORK>
\end{lstlisting}

Выведем на экран дерево каталогов и содержимое файла new\_inf.txt:

\begin{lstlisting}[language={},numbers={}]
C:\WORK>tree C:\WORK /f
%*Структура папок*)
%*Серийный номер тома*): BC51-556D
C:\WORK
%*├───*)PART1
%*│*)       new_inf.txt
%*│*)
%*└───*)PART2

C:\WORK>type PART1\new_inf.txt
%*Задание практической работы №1*)
%*Ригованов*)

C:\WORK>
\end{lstlisting}
\end{enumerate}
%%\section*{Часть № 2. Windows.}
\end{document}
